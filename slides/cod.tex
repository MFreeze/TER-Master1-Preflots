\documentclass[hyperref={},
xcolor={dvipsnames,svgnames,table},10pt]{beamer}
\usepackage{concrete} 
\usepackage{amsmath}
\usepackage{tikz}
\usepackage{algorithm}
\usepackage{algorithmic}
\usepackage[utf8]{inputenc}
\usepackage[T1]{fontenc}
\usepackage{pdfpages} 
\usepackage{multirow}
\usetheme{Warsaw}
\usepackage[frenchb]{babel}
\setlength\parindent{0pt}
%\setbeamertemplate{navigation symbols}{}



\title{TER Sur Le Problème Des Préflots}
\author{DUVILLIE Guillerme \\ Ould Mohamed Abdellahi Cheikh Mehdi}
\institute{Université Montpellier2\\ Master1-Informatique\\ Spécialité-MOCA}
\logo{\includegraphics[scale=0.2]{Logo_UM2.png}}


\begin{document}

\begin{frame}
	\titlepage 
\end{frame}

\AtBeginSubsection[]
{
	\begin{frame}<beamer>
		\frametitle{Plan}
		\tableofcontents[currentsection,currentsubsection]
	\end{frame}
}

\AtBeginSection[]
{
	\begin{frame}<beamer>
		\frametitle{Plan}
		\tableofcontents[currentsection]
	\end{frame}
}

\begin{frame}{Sommaire}
  \tableofcontents
\end{frame} 

\section{Introduction}
\subsection{Cadre du TER}
\begin{frame}{Cadre et objectifs}
	\begin{itemize}
		\item Ce TER se situe dans le cadre de Problème de flot maximum

		\item Il existe différents types d'algorithmes répondant à ce problème, certains basés sur la recherche
			de chaînes améliorantes (Edmonds-Karp, Ford Fulkerson, ...), d'autres sur les fonctions de préflots. 
			Nous nous intéresserons à la seconde catégorie.
	\end{itemize}
\end{frame}

\subsection{Motivation}
\begin{frame}{Utilité}
	
	Considérons une entreprise de création de plaques d'égoûts, devant acheminer sa production depuis
	l'usine de création au lieu de stockage. \vfill

	\begin{center}
		\includegraphics[scale=0.32]{img/etape1.png}
	\end{center}
\end{frame}

\begin{frame}{Utilité}
	Il existe différents points, par lesquels peut transiter la production.

	\begin{center}
		\includegraphics[scale=0.32]{img/etape2.png}
	\end{center}
\end{frame}

\begin{frame}{Utilité}
	Chaque passage est reliés aux autres à l'aide de routes permettant un transit plus ou moins facile
	jusqu'à destination.

	\begin{center}
		\includegraphics[scale=0.32]{img/exemple.png}
	\end{center}
\end{frame}

\begin{frame}{Utilité}
	Chaque passage est reliés aux autres à l'aide de routes permettant un transit plus ou moins facile
	jusqu'à destination.

	\begin{center}
		\includegraphics[scale=0.32]{img/exemple.png}
	\end{center}

	\begin{alertblock}{Bingo!}
		Il s'agit là d'un problème de flot maximum.
	\end{alertblock}
\end{frame}

\subsection{Définitions}
\begin{frame}{Définitions}
		Nous allons définir quelques notions qui seront utlisées par la suite
		pour le déroulement de l'algorithme de préflot et ses dérivés.
\end{frame}

\begin{frame}{La notion de flot}
	Un flot est une fonction de graphes vérifiant certaines propriétés, qui sont : \vfill \vfill
\end{frame} 

\begin{frame}{La notion de flot}
	Un flot est une fonction de graphes vérifiant certaines propriétés, qui sont : ~\\~\\
	\begin{exampleblock}{le respect de la capacité de l'arc}
		La valeur du flot sur un arc ne peut être supérieure à la capacité de l'arc et ne peut être négative
	\end{exampleblock}\vfill
\end{frame} 

\begin{frame}{La notion de flot}
	Un flot est une fonction de graphes vérifiant certaines propriétés, qui sont :~\\~\\
	\begin{block}{le respect de la capacité de l'arc}
		$$\forall (i,j) \in A, \quad 0 \leq x(i,j) \leq c(i,j)$$
	\end{block}\vfill
	\begin{exampleblock}{le respect de la loi de Kirchoff}
		Pour chaque noeud différent de la source et du puits, la quantité de flot entrant dans le noeud
		est égale à la quantité de flot sortant de ce dernier.
	\end{exampleblock}\vfill
\end{frame} 

\begin{frame}{La notion de flot}
	Un flot est une fonction de graphes vérifiant certaines propriétés, qui sont : 
	\begin{block}{le respect de la capacité de l'arc}
		$$\forall (i,j) \in A, \quad 0 \leq x(i,j) \leq c(i,j)$$
	\end{block}\vfill
	\begin{block}{le respect de la loi de Kirchoff}
		$$\forall i \in S - \{s,t\}, \quad \sum_{j \in A^+(i)} x(i,j) = \sum_{k\in A^-(i)} x(k,i)$$
	\end{block}\vfill
	\begin{exampleblock}{Le respect de la contrainte de symétrie}
		Pour chaque arc du graphe, il n'y a aucun phénomène de perte entre le sommet de départ et le
		sommet d'arrivée.
	\end{exampleblock}\vfill
\end{frame} 

\begin{frame}{La notion de flot}
	Un flot est une fonction de graphes vérifiant certaines propriétés, qui sont : 
	\begin{block}{le respect de la capacité de l'arc}
		$$\forall (i,j) \in A, \quad 0 \leq x(i,j) \leq c(i,j)$$
	\end{block}\vfill
	\begin{block}{le respect de la loi de Kirchoff}
		$$\forall i \in S - \{s,t\}, \quad \sum_{j \in A^+(i)} x(i,j) = \sum_{k\in A^-(i)} x(k,i)$$
	\end{block}\vfill
	\begin{block}{Le respect de la contrainte de symétrie}
		$$ f = \sum_{j \in A^+(s)} x(s,j) = \sum_{k\in A^- (t)} x(k,t) $$ 
	\end{block}\vfill
\end{frame} 

\begin{frame}{La notion de préflot}
	Un préflot est un flot ne respectant pas la loi de Kirchoff : la quantité de préflot sortant d'un
	noeud doit être inférieure ou égale à la quantité de préflot entrant dans celui-ci.\vfill
\end{frame}

\begin{frame}{La notion de préflot}
	Un préflot est un flot ne respectant pas la loi de Kirchoff : la quantité de préflot sortant d'un
	noeud doit être inférieure ou égale à la quantité de préflot entrant dans celui-ci.\vfill
	\begin{minipage}[c]{0.45\linewidth}
		\begin{figure}
			\begin{tikzpicture}[scale=0.7]
				\tikzset{noeud/.style={circle, draw=black, inner sep=0.1cm, minimum width=0.6cm}, fleche/.style={>=latex, ->}};
				\node[noeud] (s) at (0,0) {s};
				\node[noeud] (a) at (2, 2) {a};
				\node[noeud] (b) at (2, -2) {b};
				\node[noeud] (c) at (4.5, -2) {c};
				\node[noeud] (d) at (4.5, 2) {d};
				\node[noeud] (t) at (6.5,0) {t};

				\draw[fleche, red] (s) -- node[above left, red] {$2/2$}(a);
				\draw[fleche] (s) -- node[below left] {$0/1$}(b);
				\draw[fleche, red] (a) -- node[left,red] {$1/1$}(b);
				\draw[fleche, red] (a) -- node[above, red] {$1/3$}(d);
				\draw[fleche] (d) -- node[above left] {$0/1$}(b);
				\draw[fleche, red] (b) -- node[below, red] {$1/1$}(c);
				\draw[fleche, red] (d) -- node[above right, red] {$1/4$}(t);
				\draw[fleche, red] (c) -- node[below right, red] {$1/1$}(t);
				\draw[fleche] (c) -- node[right] {$0/2$}(d);
			\end{tikzpicture}
			\caption{Un exemple de flot}
		\end{figure}
	\end{minipage} \hfill
	\begin{minipage}[c]{0.45\linewidth}
	\end{minipage}
\end{frame}

\begin{frame}{La notion de préflot}
	Un préflot est un flot ne respectant pas la loi de Kirchoff : la quantité de préflot sortant d'un
	noeud doit être inférieure ou égale à la quantité de préflot entrant dans celui-ci.\vfill
	\begin{minipage}[c]{0.45\linewidth}
		\begin{figure}
			\begin{tikzpicture}[scale=0.7]
				\tikzset{noeud/.style={circle, draw=black, inner sep=0.1cm, minimum width=0.6cm}, fleche/.style={>=latex, ->}};
				\node[noeud] (s) at (0,0) {s};
				\node[noeud] (a) at (2, 2) {a};
				\node[noeud] (b) at (2, -2) {b};
				\node[noeud] (c) at (4.5, -2) {c};
				\node[noeud] (d) at (4.5, 2) {d};
				\node[noeud] (t) at (6.5,0) {t};

				\draw[fleche, red] (s) -- node[above left, red] {$2/2$}(a);
				\draw[fleche] (s) -- node[below left] {$0/1$}(b);
				\draw[fleche, red] (a) -- node[left,red] {$1/1$}(b);
				\draw[fleche, red] (a) -- node[above, red] {$1/3$}(d);
				\draw[fleche] (d) -- node[above left] {$0/1$}(b);
				\draw[fleche, red] (b) -- node[below, red] {$1/1$}(c);
				\draw[fleche, red] (d) -- node[above right, red] {$1/4$}(t);
				\draw[fleche, red] (c) -- node[below right, red] {$1/1$}(t);
				\draw[fleche] (c) -- node[right] {$0/2$}(d);
			\end{tikzpicture}
			\caption{Un exemple de flot}
		\end{figure}
	\end{minipage} \hfill
	\begin{minipage}[c]{0.45\linewidth}
		\begin{figure}
			\begin{tikzpicture}[scale=0.7]
				\tikzset{noeud/.style={circle, draw=black, inner sep=0.1cm, minimum width=0.6cm}, fleche/.style={>=latex, ->}};
				\node[noeud] (s) at (0,0) {s};
				\node[noeud] (a) at (2, 2) {a};
				\node[noeud] (b) at (2, -2) {b};
				\node[noeud] (c) at (4.5, -2) {c};
				\node[noeud, draw=red, red] (d) at (4.5, 2) {d};
				\node[noeud] (t) at (6.5,0) {t};

				\draw[fleche, green] (s) -- node[above left, green] {$2/2$}(a);
				\draw[fleche] (s) -- node[below left] {$0/1$}(b);
				\draw[fleche, green] (a) -- node[left,green] {$1/1$}(b);
				\draw[fleche, green] (a) -- node[above, green] {$1/3$}(d);
				\draw[fleche] (d) -- node[above left] {$0/1$}(b);
				\draw[fleche, green] (b) -- node[below, green] {$1/1$}(c);
				\draw[fleche] (d) -- node[above right] {$0/4$}(t);
				\draw[fleche, green] (c) -- node[below right, green] {$1/1$}(t);
				\draw[fleche] (c) -- node[right] {$0/2$}(d);
			\end{tikzpicture}
			\caption{Un exemple de préflot}
		\end{figure}
	\end{minipage}
\end{frame}

\begin{frame}{Noeud Actif}
	Un noeud dont la quantité de préflot entrant est supérieure à la quantité de préflot sortant est
	dit actif, la différence est appelée excédent et est notée $e$. \vfill
\end{frame}

\begin{frame}{Noeud Actif}
	Un noeud dont la quantité de préflot entrant est supérieure à la quantité de préflot sortant est
	dit actif, la différence est appelée excédent et est notée $e$.

	\begin{center}
		\begin{tikzpicture}[scale=0.65]
			\tikzset{noeud/.style={circle, draw=black, inner sep=0.1cm, minimum width=0.6cm}, fleche/.style={>=latex, ->}};
			\node[noeud] (s) at (0,0) {s};
			\node[noeud] (a) at (2, 2) {a};
			\node[noeud] (b) at (2, -2) {b};
			\node[noeud] (c) at (4.5, -2) {c};
			\node[noeud, draw=red, red] (d) at (4.5, 2) {d};
			\node[noeud] (t) at (6.5,0) {t};

			\draw[fleche, green] (s) -- node[above left, green] {$2/2$}(a);
			\draw[fleche] (s) -- node[below left] {$0/1$}(b);
			\draw[fleche, green] (a) -- node[left,green] {$1/1$}(b);
			\draw[fleche, green] (a) -- node[above, green] {$1/3$}(d);
			\draw[fleche] (d) -- node[above left] {$0/1$}(b);
			\draw[fleche, green] (b) -- node[below, green] {$1/1$}(c);
			\draw[fleche] (d) -- node[above right] {$0/4$}(t);
			\draw[fleche, green] (c) -- node[below right, green] {$1/1$}(t);
			\draw[fleche] (c) -- node[right] {$0/2$}(d);
		\end{tikzpicture}
	\end{center}

	\begin{alertblock}{Remarque}
		Ici $d$ est actif et $e(d) = 1$
	\end{alertblock}
\end{frame}

\begin{frame}{Le réseau résiduel}
 	Soit un graphe $G(S, A)$, un flot (ou préflot) $f$ et $i,j$ deux sommets de $G$. On dira que
	l'arête $(i,j)$ appartiendra au réseau résiduel $A_f$ si et seulement si la capacité résiduelle
	est non nulle.

	\begin{alertblock}{Autrement dit}
	 	$$ r(i,j) = c(i,j) - x(i,j) > 0 $$ 
	\end{alertblock}

	On appelle alors $r(i,j)$ la capacité résiduelle de l'arête $(i,j)$. 
	\vfill
\end{frame}

\begin{frame}{Le réseau résiduel}
 	Soit un graphe $G(S, A)$, un flot (ou préflot) $f$ et $i,j$ deux sommets de $G$. On dira que
	l'arête $(i,j)$ appartiendra au réseau résiduel $A_f$ si et seulement si la capacité résiduelle
	est non nulle.

	\begin{alertblock}{Autrement dit}
	 	$$ r(i,j) = c(i,j) - x(i,j) > 0 $$ 
	\end{alertblock}


	On appelle alors $r(i,j)$ la capacité résiduelle de l'arête $(i,j)$. 

	\begin{block}{Note}
		Si $x(i,j) > 0$ alors $(j,i)$ appartient au réseau résiduel avec une capacité résiduelle $r(j,i) = x(i,j)$.
	\end{block}
\end{frame}

\begin{frame}{Le réseau résiduel}
	Pour le flot suivant, ... \vfill
	\begin{minipage}[c]{0.45\linewidth}
		\begin{figure}
			\begin{tikzpicture}[scale=0.7]
				\tikzset{noeud/.style={circle, draw=black, inner sep=0.1cm, minimum width=0.6cm}, fleche/.style={>=latex, ->}};
				\node[noeud] (s) at (0,0) {s};
				\node[noeud] (a) at (2, 2) {a};
				\node[noeud] (b) at (2, -2) {b};
				\node[noeud] (c) at (4.5, -2) {c};
				\node[noeud] (d) at (4.5, 2) {d};
				\node[noeud] (t) at (6.5,0) {t};

				\draw[fleche, red] (s) -- node[above left, red] {$2/2$}(a);
				\draw[fleche] (s) -- node[below left] {$0/1$}(b);
				\draw[fleche, red] (a) -- node[left,red] {$1/1$}(b);
				\draw[fleche, red] (a) -- node[above, red] {$1/3$}(d);
				\draw[fleche] (d) -- node[above left] {$0/1$}(b);
				\draw[fleche, red] (b) -- node[below, red] {$1/1$}(c);
				\draw[fleche, red] (d) -- node[above right, red] {$1/4$}(t);
				\draw[fleche, red] (c) -- node[below right, red] {$1/1$}(t);
				\draw[fleche] (c) -- node[right] {$0/2$}(d);
			\end{tikzpicture}
			\caption{Un exemple de flot}
		\end{figure}
	\end{minipage} \hfill
	\begin{minipage}[c]{0.45\linewidth}
	\end{minipage}
\end{frame}

\begin{frame}{Le réseau résiduel}
	$\dots$ on a le réseau résiduel suivant : \vfill
	\begin{minipage}[c]{0.45\linewidth}
		\begin{figure}
			\begin{tikzpicture}[scale=0.7]
				\tikzset{noeud/.style={circle, draw=black, inner sep=0.1cm, minimum width=0.6cm}, fleche/.style={>=latex, ->}};
				\node[noeud] (s) at (0,0) {s};
				\node[noeud] (a) at (2, 2) {a};
				\node[noeud] (b) at (2, -2) {b};
				\node[noeud] (c) at (4.5, -2) {c};
				\node[noeud] (d) at (4.5, 2) {d};
				\node[noeud] (t) at (6.5,0) {t};

				\draw[fleche, red] (s) -- node[above left, red] {$2/2$}(a);
				\draw[fleche] (s) -- node[below left] {$0/1$}(b);
				\draw[fleche, red] (a) -- node[left,red] {$1/1$}(b);
				\draw[fleche, red] (a) -- node[above, red] {$1/3$}(d);
				\draw[fleche] (d) -- node[above left] {$0/1$}(b);
				\draw[fleche, red] (b) -- node[below, red] {$1/1$}(c);
				\draw[fleche, red] (d) -- node[above right, red] {$1/4$}(t);
				\draw[fleche, red] (c) -- node[below right, red] {$1/1$}(t);
				\draw[fleche] (c) -- node[right] {$0/2$}(d);
			\end{tikzpicture}
			\caption{Un exemple de flot}
		\end{figure}
	\end{minipage} \hfill
	\begin{minipage}[c]{0.45\linewidth}
		\begin{figure}
			\begin{tikzpicture}[scale=0.7]
				\tikzset{noeud/.style={circle, draw=black, inner sep=0.1cm, minimum width=0.6cm}, fleche/.style={>=latex, ->}};
				\node[noeud] (s) at (0,0) {s};
				\node[noeud] (a) at (2, 2) {a};
				\node[noeud] (b) at (2, -2) {b};
				\node[noeud] (c) at (4.5, -2) {c};
				\node[noeud] (d) at (4.5, 2) {d};
				\node[noeud] (t) at (6.5,0) {t};

				\draw[fleche, purple] (a) -- node[above left, purple] {$2$}(s);
				\draw[fleche] (s) -- node[below left] {$1$}(b);
				\draw[fleche, purple] (b) -- node[left, purple] {$1$}(a);
				\draw[fleche] (a) to[out=15, in=165]  node[above] {$2$}(d);
				\draw[fleche, purple] (d) to[out=195, in=345]  node[below, purple] {$1$}(a);
				\draw[fleche] (d) -- node[below right] {$1$}(b);
				\draw[fleche, purple] (c) -- node[below, purple] {$1$}(b);
				\draw[fleche] (d) to[out=295, in=145] node[below left] {$3$}(t);
				\draw[fleche, purple] (t) to[out=115, in=335] node[above right, purple] {$1$}(d);
				\draw[fleche, purple] (t) -- node[below right, purple] {$1$}(c);
				\draw[fleche] (c) -- node[right] {$2$}(d);
			\end{tikzpicture}
			\caption{Le réseau résiduel associé}
		\end{figure}
	\end{minipage}
\end{frame}

\begin{frame}{La fonction de distance}
	Une fonction de distance est une fonction représentant le plus court chemin en nombre d'arcs d'un
	noeud au puits, vérifiant les propriétés suivantes :\vfill
\end{frame}

\begin{frame}{La fonction de distance}
	Une fonction de distance est une fonction représentant le plus court chemin en nombre d'arcs d'un
	noeud au puits, vérifiant les propriétés suivantes : \vfill
	\begin{exampleblock}{Distance de la source}
		la distance de la source est égale au nombre de sommets du graphe
	\end{exampleblock} \vfill
\end{frame}

\begin{frame}{La fonction de distance}
	Une fonction de distance est une fonction représentant le plus court chemin en nombre d'arcs d'un
	noeud au puits, vérifiant les propriétés suivantes : \vfill
	\begin{block}{Propriétés}
		$$ d(s) = |S| $$
	\end{block} \vfill
	\begin{exampleblock}{Distance du puits}
		la distance du puits est égale à zéro
	\end{exampleblock} \vfill
\end{frame}

\begin{frame}{La fonction de distance}
	Une fonction de distance est une fonction représentant la distance d'un noeud au puits, vérifiant
	les propriétés suivantes : \vfill
	\begin{block}{Propriétés}
		$$ d(s) = |S| $$
		$$ d(t) = 0 $$
	\end{block} \vfill
	\begin{exampleblock}{Distances des sommets}
		si $(i,j)$ appartient au réseau résiduel, la distance du sommet $i$ est égale à la distance du sommet $i$ plus un
	\end{exampleblock}
\end{frame}

\begin{frame}{La fonction de distance}
	Une fonction de distance est une fonction représentant la distance d'un noeud au puits, vérifiant
	les propriétés suivantes : \vfill
	\begin{block}{Propriétés}
		$$ d(s) = |S| $$
		$$ d(t) = 0 $$
		$$ \forall (i,j) \in A_f,\quad d(i) = d(j) + 1 $$
	\end{block}
\end{frame}

\begin{frame}{Conséquences sur le réseau résiduel}
	Le réseau résiduel associé au flot suivant : \vfill
	\begin{minipage}[c]{0.45\linewidth}
		\begin{figure}
			\begin{tikzpicture}[scale=0.7]
				\tikzset{noeud/.style={circle, draw=black, inner sep=0.1cm, minimum width=0.6cm}, fleche/.style={>=latex, ->}};
				\node[noeud] (s) at (0,0) {s};
				\node[noeud] (a) at (2, 2) {a};
				\node[noeud] (b) at (2, -2) {b};
				\node[noeud] (c) at (4.5, -2) {c};
				\node[noeud] (d) at (4.5, 2) {d};
				\node[noeud] (t) at (6.5,0) {t};

				\draw[fleche, red] (s) -- node[above left, red] {$2/2$}(a);
				\draw[fleche] (s) -- node[below left] {$0/1$}(b);
				\draw[fleche, red] (a) -- node[left,red] {$1/1$}(b);
				\draw[fleche, red] (a) -- node[above, red] {$1/3$}(d);
				\draw[fleche] (d) -- node[above left] {$0/1$}(b);
				\draw[fleche, red] (b) -- node[below, red] {$1/1$}(c);
				\draw[fleche, red] (d) -- node[above right, red] {$1/4$}(t);
				\draw[fleche, red] (c) -- node[below right, red] {$1/1$}(t);
				\draw[fleche] (c) -- node[right] {$0/2$}(d);
			\end{tikzpicture}
			\caption{Un exemple de flot}
		\end{figure}
	\end{minipage} \hfill
	\begin{minipage}[c]{0.45\linewidth}
	\end{minipage}
\end{frame}

\begin{frame}{Conséquences sur le réseau résiduel}
	Le réseau résiduel associé au flot suivant, devient : \vfill
	\begin{minipage}[c]{0.45\linewidth}
		\begin{figure}
			\begin{tikzpicture}[scale=0.7]
				\tikzset{noeud/.style={circle, draw=black, inner sep=0.1cm, minimum width=0.6cm}, fleche/.style={>=latex, ->}};
				\node[noeud] (s) at (0,0) {s};
				\node[noeud] (a) at (2, 2) {a};
				\node[noeud] (b) at (2, -2) {b};
				\node[noeud] (c) at (4.5, -2) {c};
				\node[noeud] (d) at (4.5, 2) {d};
				\node[noeud] (t) at (6.5,0) {t};

				\draw[fleche, red] (s) -- node[above left, red] {$2/2$}(a);
				\draw[fleche] (s) -- node[below left] {$0/1$}(b);
				\draw[fleche] (a) -- node[left] {$0/1$}(b);
				\draw[fleche, red] (a) -- node[above, red] {$1/3$}(d);
				\draw[fleche] (d) -- node[above left] {$0/1$}(b);
				\draw[fleche] (b) -- node[below] {$0/1$}(c);
				\draw[fleche, red] (d) -- node[above right, red] {$1/4$}(t);
				\draw[fleche] (c) -- node[below right] {$0/1$}(t);
				\draw[fleche] (c) -- node[right] {$0/2$}(d);
			\end{tikzpicture}
			\caption{Un exemple de flot}
		\end{figure}
	\end{minipage} \hfill
	\begin{minipage}[c]{0.45\linewidth}
		\begin{figure}
			\begin{tikzpicture}[scale=0.7]
				\tikzset{noeud/.style={circle, draw=black, inner sep=0.1cm, minimum width=0.6cm}, fleche/.style={>=latex, ->}};
				\node[noeud] (s) at (0,0) {s};
				\node[noeud] (a) at (2, 2) {a};
				\node[noeud] (b) at (2, -2) {b};
				\node[noeud] (c) at (4.5, -2) {c};
				\node[noeud] (d) at (4.5, 2) {d};
				\node[noeud] (t) at (6.5,0) {t};

				\draw[fleche] (s) -- node[below left] {$1$}(b);
				\draw[fleche] (a) to node[above] {$2$}(d);
				\draw[fleche] (d) to node[below left] {$3$}(t);
				\draw[fleche] (c) -- node[below right] {$1$}(t);
				\draw[fleche] (b) -- node[below] {$1$}(c);
			\end{tikzpicture}
			\caption{Le réseau résiduel associé}
		\end{figure}
	\end{minipage}
\end{frame}

\begin{frame}{Introduction aux opérations}
	Considérons le graphe suivant :\vfill

	\begin{figure}
	\begin{center}
		\begin{tikzpicture}[scale=0.6]
			\tikzset{noeud/.style={circle, draw=black, text centered,minimum size=16pt, inner sep=0pt},
			fleche/.style={thick}};

			\node[noeud] (t) at (14, 0) {$t$};
			\node[noeud] (s) at (2, 0) {$s$};

			\foreach \x in {2, 3, 4, 5}{
				\node[noeud] (\x) at (2*\x, 0) {$\x$}; 
			}

			\foreach \y/\ytext in {3.5/6, 1.7/7, -1.7/55, -3.5/56}{
				\node[noeud] (\ytext) at (12, \y) {$\ytext$};
				\draw[fleche] (5) --node[right] {$1$} (\ytext) --node[left] {$1$} (t);
			}

			\node (point) at (12, 0) {$\vdots$};

			\draw[fleche] (s) --node[above] {$\infty$} (2) --node[above] {$\infty$} (3) --node[above] {$\infty$}
			(4) --node[above] {$\infty$} (5) ;

		\end{tikzpicture}
	\end{center}
\end{figure} \vfill
\end{frame}

\begin{frame}{Introduction aux opérations}
	Considérons le graphe suivant :\vfill

	\begin{figure}
	\begin{center}
		\begin{tikzpicture}[scale=0.6]
			\tikzset{noeud/.style={circle, draw=black, text centered,minimum size=16pt, inner sep=0pt},
			fleche/.style={thick}};

			\node[noeud] (t) at (14, 0) {$t$};
			\node[noeud] (s) at (2, 0) {$s$};

			\foreach \x in {2, 3, 4, 5}{
				\node[noeud] (\x) at (2*\x, 0) {$\x$}; 
			}

			\foreach \y/\ytext in {3.5/6, 1.7/7, -1.7/55, -3.5/56}{
				\node[noeud] (\ytext) at (12, \y) {$\ytext$};
				\draw[fleche] (5) --node[right] {$1$} (\ytext) --node[left] {$1$} (t);
			}

			\node (point) at (12, 0) {$\vdots$};

			\draw[fleche] (s) --node[above] {$\infty$} (2) --node[above] {$\infty$} (3) --node[above] {$\infty$}
			(4) --node[above] {$\infty$} (5) ;

		\end{tikzpicture}
	\end{center}
\end{figure} \vfill
\begin{alertblock}{Constat}
Il met en défaut les algorithmes de recherche de chaînes améliorantes.
\end{alertblock}
\end{frame}

\begin{frame}{Poussage}
	\textbf{Poussage :} L'action de faire transiter le préflot d'un noeud $i$ à un noeud $j$ est appelée
	opération de poussage.\\
\end{frame}

\begin{frame}{Poussage}
	\textbf{Poussage :} L'action de faire transiter le préflot d'un noeud $i$ à un noeud $j$ est appelée
	opération de poussage.\\
	\begin{alertblock}{Condition}
		On ne peut pas appliquer un poussage sur l'arrête $(i,j)$ que si le sommet $i$ est
		actif, $C_f$ $(i,j)>0$ et $d(i) = d(j)$ + 1.\\
	\end{alertblock}
\end{frame}

\begin{frame}{Application des poussages}
	Reprenons le graphe précédent :\vfill

	\begin{figure}
	\begin{center}
		\begin{tikzpicture}[scale=0.6]
			\tikzset{noeud/.style={circle, draw=black, text centered,minimum size=16pt, inner sep=0pt},
			fleche/.style={thick}};

			\node[noeud] (t) at (14, 0) {$t$};
			\node[noeud] (s) at (2, 0) {$s$};

			\foreach \x in {2, 3, 4, 5}{
				\node[noeud] (\x) at (2*\x, 0) {$\x$}; 
			}

			\foreach \y/\ytext in {3.5/6, 1.7/7, -1.7/55, -3.5/56}{
				\node[noeud] (\ytext) at (12, \y) {$\ytext$};
				\draw[fleche] (5) --node[right] {$1$} (\ytext) --node[left] {$1$} (t);
			}

			\node (point) at (12, 0) {$\vdots$};

			\draw[fleche] (s) --node[above] {$\infty$} (2);
			\draw[fleche] (2) --node[above] {$\infty$} (3);
			\draw[fleche] (3) --node[above] {$\infty$} (4);
			\draw[fleche] (4) --node[above] {$\infty$} (5) ;

		\end{tikzpicture}
	\end{center}
\end{figure} \vfill
\end{frame}

\begin{frame}{Application des poussages}
	Reprenons le graphe précédent :\vfill

	\begin{figure}
	\begin{center}
		\begin{tikzpicture}[scale=0.6]
			\tikzset{noeud/.style={circle, draw=black, text centered,minimum size=16pt, inner sep=0pt},
			fleche/.style={thick}};

			\node[noeud] (t) at (14, 0) {$t$};
			\node[noeud] (s) at (2, 0) {$s$};

			\foreach \x in {2, 3, 4, 5}{
				\node[noeud] (\x) at (2*\x, 0) {$\x$}; 
			}

			\foreach \y/\ytext in {3.5/6, 1.7/7, -1.7/55, -3.5/56}{
				\node[noeud] (\ytext) at (12, \y) {$\ytext$};
				\draw[fleche] (5) --node[right] {$1$} (\ytext) --node[left] {$1$} (t);
			}

			\node (point) at (12, 0) {$\vdots$};

			\draw[fleche, red] (s) --node[above, red] {$\infty$} (2);
			\draw[fleche] (2) --node[above] {$\infty$} (3);
			\draw[fleche] (3) --node[above] {$\infty$} (4);
			\draw[fleche] (4) --node[above] {$\infty$} (5) ;

		\end{tikzpicture}
	\end{center}
\end{figure} \vfill
\end{frame}

\begin{frame}{Application des poussages}
	Reprenons le graphe précédent :\vfill

	\begin{figure}
	\begin{center}
		\begin{tikzpicture}[scale=0.6]
			\tikzset{noeud/.style={circle, draw=black, text centered,minimum size=16pt, inner sep=0pt},
			fleche/.style={thick}};

			\node[noeud] (t) at (14, 0) {$t$};
			\node[noeud] (s) at (2, 0) {$s$};

			\foreach \x in {2, 3, 4, 5}{
				\node[noeud] (\x) at (2*\x, 0) {$\x$}; 
			}

			\foreach \y/\ytext in {3.5/6, 1.7/7, -1.7/55, -3.5/56}{
				\node[noeud] (\ytext) at (12, \y) {$\ytext$};
				\draw[fleche] (5) --node[right] {$1$} (\ytext) --node[left] {$1$} (t);
			}

			\node (point) at (12, 0) {$\vdots$};

			\draw[fleche, red] (s) --node[above, red] {$\infty$} (2);
			\draw[fleche, red] (2) --node[above, red] {$\infty$} (3);
			\draw[fleche] (3) --node[above] {$\infty$} (4);
			\draw[fleche] (4) --node[above] {$\infty$} (5) ;

		\end{tikzpicture}
	\end{center}
\end{figure} \vfill
\end{frame}

\begin{frame}{Application des poussages}
	Reprenons le graphe précédent :\vfill

	\begin{figure}
	\begin{center}
		\begin{tikzpicture}[scale=0.6]
			\tikzset{noeud/.style={circle, draw=black, text centered,minimum size=16pt, inner sep=0pt},
			fleche/.style={thick}};

			\node[noeud] (t) at (14, 0) {$t$};
			\node[noeud] (s) at (2, 0) {$s$};

			\foreach \x in {2, 3, 4, 5}{
				\node[noeud] (\x) at (2*\x, 0) {$\x$}; 
			}

			\foreach \y/\ytext in {3.5/6, 1.7/7, -1.7/55, -3.5/56}{
				\node[noeud] (\ytext) at (12, \y) {$\ytext$};
				\draw[fleche] (5) --node[right] {$1$} (\ytext) --node[left] {$1$} (t);
			}

			\node (point) at (12, 0) {$\vdots$};

			\draw[fleche, red] (s) --node[above, red] {$\infty$} (2);
			\draw[fleche, red] (2) --node[above, red] {$\infty$} (3);
			\draw[fleche, red] (3) --node[above, red] {$\infty$} (4);
			\draw[fleche] (4) --node[above] {$\infty$} (5) ;

		\end{tikzpicture}
	\end{center}
\end{figure} \vfill
\end{frame}

\begin{frame}{Application des poussages}
	Reprenons le graphe précédent :\vfill

	\begin{figure}
	\begin{center}
		\begin{tikzpicture}[scale=0.6]
			\tikzset{noeud/.style={circle, draw=black, text centered,minimum size=16pt, inner sep=0pt},
			fleche/.style={thick}};

			\node[noeud] (t) at (14, 0) {$t$};
			\node[noeud] (s) at (2, 0) {$s$};

			\foreach \x in {2, 3, 4, 5}{
				\node[noeud] (\x) at (2*\x, 0) {$\x$}; 
			}

			\foreach \y/\ytext in {3.5/6, 1.7/7, -1.7/55, -3.5/56}{
				\node[noeud] (\ytext) at (12, \y) {$\ytext$};
				\draw[fleche] (5) --node[right] {$1$} (\ytext) --node[left] {$1$} (t);
			}

			\node (point) at (12, 0) {$\vdots$};

			\draw[fleche, red] (s) --node[above, red] {$\infty$} (2);
			\draw[fleche, red] (2) --node[above, red] {$\infty$} (3);
			\draw[fleche, red] (3) --node[above, red] {$\infty$} (4);
			\draw[fleche, red] (4) --node[above, red] {$\infty$} (5) ;

		\end{tikzpicture}
	\end{center}
\end{figure} \vfill
\end{frame}

\begin{frame}{Application des poussages}
	Reprenons le graphe précédent :\vfill

	\begin{figure}
	\begin{center}
		\begin{tikzpicture}[scale=0.6]
			\tikzset{noeud/.style={circle, draw=black, text centered,minimum size=16pt, inner sep=0pt},
			fleche/.style={thick}};

			\node[noeud] (t) at (14, 0) {$t$};
			\node[noeud] (s) at (2, 0) {$s$};

			\foreach \x in {2, 3, 4, 5}{
				\node[noeud] (\x) at (2*\x, 0) {$\x$}; 
			}

			\foreach \y/\ytext in {3.5/6, 1.7/7, -1.7/55, -3.5/56}{
				\node[noeud] (\ytext) at (12, \y) {$\ytext$};
				\draw[fleche, red] (5) --node[right, red] {$1$} (\ytext) --node[left] {$1$} (t);
			}

			\node (point) at (12, 0) {$\vdots$};

			\draw[fleche, red] (s) --node[above, red] {$\infty$} (2);
			\draw[fleche, red] (2) --node[above, red] {$\infty$} (3);
			\draw[fleche, red] (3) --node[above, red] {$\infty$} (4);
			\draw[fleche, red] (4) --node[above, red] {$\infty$} (5) ;

		\end{tikzpicture}
	\end{center}
\end{figure} \vfill
\end{frame}

\begin{frame}{Application des poussages}
	Reprenons le graphe précédent :\vfill

	\begin{figure}
	\begin{center}
		\begin{tikzpicture}[scale=0.6]
			\tikzset{noeud/.style={circle, draw=black, text centered,minimum size=16pt, inner sep=0pt},
			fleche/.style={thick}};

			\node[noeud] (t) at (14, 0) {$t$};
			\node[noeud] (s) at (2, 0) {$s$};

			\foreach \x in {2, 3, 4, 5}{
				\node[noeud] (\x) at (2*\x, 0) {$\x$}; 
			}

			\foreach \y/\ytext in {3.5/6, 1.7/7, -1.7/55, -3.5/56}{
				\node[noeud] (\ytext) at (12, \y) {$\ytext$};
				\draw[fleche, red] (5) --node[right, red] {$1$} (\ytext) --node[left] {$1$} (t);
			}

			\node (point) at (12, 0) {$\vdots$};

			\draw[fleche, red] (s) --node[above, red] {$\infty$} (2);
			\draw[fleche, red] (2) --node[above, red] {$\infty$} (3);
			\draw[fleche, red] (3) --node[above, red] {$\infty$} (4);
			\draw[fleche, red] (4) --node[above, red] {$\infty$} (5) ;

		\end{tikzpicture}
	\end{center}
\end{figure} \vfill
\begin{exampleblock}{Et maintenant...}
	Que faire?
\end{exampleblock}
\end{frame}

\begin{frame}{Réétiquetage}
	\textbf{Réétiquetage :} Le ré-étiquetage consiste à réévaluer la distance du noeud actif au
	noeud puits afin qu'il soit possible, après cette opération, d'effectuer une opération de
	poussage depuis ce noeud.
\end{frame}

\begin{frame}{Réétiquetage}
	\textbf{Réétiquetage :} Le ré-étiquetage consiste à réévaluer la distance du noeud actif au
	noeud puits afin qu'il soit possible, après cette opération, d'effectuer une opération de
	poussage depuis ce noeud.
	\begin{alertblock}{Condition}
		On ne peut pas réétiqueter que si $i$ est actif et, pour tout $j \in S$ tel
		que $(i,j) \in A_f$. On a $d (i)\leq d (j)$. 
	\end{alertblock}
\end{frame}

\section{Présentation de l'Algorithme de préflot-Générique}
\subsection{Principe}

\begin{frame}{Première approche}
	Le graphe est assimilé à un réseau de tuyaux de capacité donnée.\vfill
\end{frame}

\begin{frame}{Première approche}
	Le graphe est assimilé à un réseau de tuyaux de capacité donnée.\vfill
	Chaque noeud possède un réservoir de capacité infinie.\vfill
\end{frame}

\begin{frame}{Première approche}
	Le graphe est assimilé à un réseau de tuyaux de capacité donnée.\vfill
	Chaque noeud possède un réservoir de capacité infinie.\vfill
	Seule la source possède un réservoir plein. \vfill
\end{frame}

\begin{frame}{Première approche}
	Le graphe est assimilé à un réseau de tuyaux de capacité donnée.\vfill
	Chaque noeud possède un réservoir de capacité infinie.\vfill
	Seule la source possède un réservoir plein. \vfill
	À chaque noeud est associée une hauteur.\vfill
\end{frame}

\begin{frame}{Première approche}
	Le graphe est assimilé à un réseau de tuyaux de capacité donnée.\vfill
	Chaque noeud possède un réservoir de capacité infinie.\vfill
	Seule la source possède un réservoir plein. \vfill
	À chaque noeud est associée une hauteur.\vfill
	Deux actions possibles : \begin{enumerate}
		\item vidange du réservoir
	\end{enumerate} \vfill
\end{frame}

\begin{frame}{Première approche}
	Le graphe est assimilé à un réseau de tuyaux de capacité donnée.\vfill
	Chaque noeud possède un réservoir de capacité infinie.\vfill
	Seule la source possède un réservoir plein. \vfill
	À chaque noeud est associée une hauteur.\vfill
	Deux actions possibles : \begin{enumerate}
		\item vidange du réservoir
		\item augmentation de la hauteur du réservoir
	\end{enumerate} \vfill
\end{frame}

\begin{frame}{Première approche}
	Le graphe est assimilé à un réseau de tuyaux de capacité donnée.\vfill
	Chaque noeud possède un réservoir de capacité infinie.\vfill
	Seule la source possède un réservoir plein. \vfill
	À chaque noeud est associée une hauteur.\vfill
	Deux actions possibles : \begin{enumerate}
		\item vidange du réservoir
		\item augmentation de la hauteur du réservoir
	\end{enumerate} \vfill
	On commence par vidanger la source, puis on cherche à vidanger les noeuds dont le réservoir n'est
	pas vide. \vfill
\end{frame}

\begin{frame}{Première approche}
	Le graphe est assimilé à un réseau de tuyaux de capacité donnée.\vfill
	Chaque noeud possède un réservoir de capacité infinie.\vfill
	Seule la source possède un réservoir plein. \vfill
	À chaque noeud est associée une hauteur.\vfill
	Deux actions possibles : \begin{enumerate}
		\item vidange du réservoir
		\item augmentation de la hauteur du réservoir
	\end{enumerate} \vfill
	On commence par vidanger la source, puis on cherche à vidanger les noeuds dont le réservoir n'est
	pas vide. \vfill
	Si on ne peut vidanger aucun réservoir et qu'il reste des réservoirs non vides, on augmente la
	hauteur de ces derniers. \vfill
\end{frame}

\begin{frame}{Approche formelle}
	L'algorithme examine les noeuds actifs.\vfill
\end{frame}

\begin{frame}{Approche formelle}
	L'algorithme examine les noeuds actifs.\vfill
	Il cherche à réduire l'excédent de flot en l'envoyant vers un noeud voisin dans le graphe
	résiduel (Poussage).\vfill
\end{frame}

\begin{frame}{Approche formelle}
	L'algorithme examine les noeuds actifs.\vfill
	Il cherche à réduire l'excédent de flot en l'envoyant vers un noeud voisin dans le graphe
	résiduel (Poussage).\vfill
	S'il n'existe aucun voisin, on augmente la distance du noeud au puits (Ré-étiquetage).\vfill
\end{frame}

\begin{frame}{Approche formelle}
	L'algorithme examine les noeuds actifs.\vfill
	Il cherche à réduire l'excédent de flot en l'envoyant vers un noeud voisin dans le graphe
	résiduel (Poussage).\vfill
	S'il n'existe aucun voisin, on augmente la distance du noeud au puits (Ré-étiquetage).\vfill
	Lorsque plus aucun noeud est actif, le préflot restant est un flot maximum.\vfill
\end{frame}

\subsection{Les procédures}

\begin{frame}{Initialisation}
	Consiste à calculer les distances de chacun des noeuds de façon à obtenir une fonction de distance
	valide.\vfill
\end{frame}

\begin{frame}{Initialisation}
	Consiste à calculer les distances de chacun des noeuds de façon à obtenir une fonction de distance
	valide.\vfill
	Procédure d'initialisation : 
	\begin{algorithmic}[1]
			\FOR{$i\ \in\ S$}
				\STATE Calculer la distance $d(i)$ en nombre d'arêtes de $i$ à $t$  
			\ENDFOR
			\FOR{$a$ $\in$ $A(s)$}
				\STATE $x(a)\ \leftarrow$ $c(a)$  
			\ENDFOR
			\STATE $d(s)\ \leftarrow |S|$ 
	\end{algorithmic}
\end{frame}

\begin{frame}{Examen des noeuds}
	Consiste à sélectionner un noeud actif et à effectuer un poussage si possible, ou un ré-étiquetage
	sinon. \vfill
\end{frame}

\begin{frame}{Examen des noeuds}
	Consiste à sélectionner un noeud actif et à effectuer un poussage si possible, ou un ré-étiquetage
	sinon. \vfill

	Procédure Pousser-Réétiqueter :
	\begin{algorithmic}[1]
		\IF {$e(i) > 0$}
		\IF {$\exists j$ tel que $c(i,j) > 0$ \AND $d(i) = d(j) + 1$}
				\STATE $\delta \leftarrow \min(e(i), c(i,j))$
				\STATE $f(i,j) \leftarrow f(i,j) + \delta$
				\STATE $e(i) \leftarrow e(i) - \delta$
				\STATE $e(j) \leftarrow e(j) + \delta$
			\ELSE
				\STATE $d(i) \leftarrow 1 + \min\{ d(j) / (i,j) \in A_f\}$
			\ENDIF
		\ENDIF
	\end{algorithmic}\vfill
\end{frame}

\begin{frame}{La procédure principale}
	Consiste à initialiser le problème, et parcourir les noeuds pour l'application du poussage et
	ré-étiquetage.\vfill

	Algorithme Générique :
	\begin{algorithmic}[1]
			\STATE Initialisation() \\
			\WHILE{Il existe un noeud actif $i$}
				\STATE Pousser-Réétiqueter(i) 
			\ENDWHILE
	\end{algorithmic}\vfill
\end{frame}

\subsection{Exemple}

\begin{frame}{Sur le graphe de départ}
	Reprenons notre entreprise $\dots$\vfill
	\begin{center}
		\includegraphics[scale=0.32]{img/exemple.png}
	\end{center}
\end{frame}

\begin{frame}{Sur le graphe de départ}
	$\dots$ et associons lui un graphe.\vfill
	\begin{center}
		\begin{tikzpicture}
			\tikzset{noeud/.style={circle, draw=black, inner sep=0.1cm, minimum width=0.6cm}, fleche/.style={>=latex, ->}};

			\node[noeud] (s) at (   0,  0) {s};
			\node[noeud] (a) at (   3,  2) {a};
			\node[noeud] (b) at (   3, -2) {b};
			\node[noeud] (c) at ( 4.5,  0) {c};
			\node[noeud] (d) at (   6,  2) {d};
			\node[noeud] (e) at (   6, -2) {e};
			\node[noeud] (t) at (   9,  0) {t};

			\draw[fleche] (s) to[out= 45, in=205] node [above  left] {10} (a);
			\draw[fleche] (a) to[out=235, in= 15] node [below right] {10} (s);
			\draw[fleche] (s) to                  node [below  left] { 1} (b);
			\draw[fleche] (a) to                  node [above      ] { 6} (d);
			\draw[fleche] (a) to[out=285, in=145] node [below  left] { 7} (c);
			\draw[fleche] (c) to[out=105, in=325] node [above right] { 7} (a);
			\draw[fleche] (c) to                  node [above  left] { 2} (b);
			\draw[fleche] (b) to                  node [below      ] { 5} (e);
			\draw[fleche] (c) to                  node [above right] { 2} (e);
			\draw[fleche] (d) to[out=255, in=105] node [left       ] { 3} (e);
			\draw[fleche] (e) to[out= 75, in=285] node [right      ] { 3} (d);
			\draw[fleche] (d) to                  node [above right] { 2} (t);
			\draw[fleche] (e) to                  node [below right] { 2} (t);
		\end{tikzpicture}
	\end{center}
\end{frame}

\begin{frame}{Initialisation}
	\begin{minipage}[c]{0.3\linewidth}
		\begin{tikzpicture}[scale=0.55]
			\tikzset{noeud/.style={circle, draw=black, inner sep=0.1cm, minimum width=0.6cm}, fleche/.style={>=latex, ->}};

			\node[noeud] (s) at (   0,    0) {s};
			\node[noeud] (a) at (  -2,   -3) {a};
			\node[noeud] (b) at (   2,   -3) {b};
			\node[noeud] (c) at (   0, -5.5) {c};
			\node[noeud] (d) at (  -2,   -8) {d};
			\node[noeud] (e) at (   2,   -8) {e};
			\node[noeud] (t) at (   0,  -11) {t};

			\draw[fleche] (s) to                  node [above  left] {0} (a);
			\draw[fleche] (s) to                  node [above right] {0} (b);
			\draw[fleche] (a) to                  node [left       ] {0} (d);
			\draw[fleche] (a) to[out=300, in=170] node [below  left] {0} (c);
			\draw[fleche] (c) to[out=120, in=345] node [above right] {0} (a);
			\draw[fleche] (c) to                  node [above  left] {0} (b);
			\draw[fleche] (b) to                  node [right      ] {0} (e);
			\draw[fleche] (c) to                  node [above right] {0} (e);
			\draw[fleche] (d) to[out= 15, in=165] node [above      ] {0} (e);
			\draw[fleche] (e) to[out=195, in=345] node [below      ] {0} (d);
			\draw[fleche] (d) to                  node [below  left] {0} (t);
			\draw[fleche] (e) to                  node [below right] {0} (t);
		\end{tikzpicture}
	\end{minipage}\hfill
	\begin{minipage}[c]{0.3\linewidth}
		\begin{tikzpicture}[scale=0.55]
			\tikzset{noeud/.style={circle, draw=black, inner sep=0.1cm, minimum width=0.6cm}, fleche/.style={>=latex, ->}};

			\node[noeud] (s) at (   0,    0) {s};
			\node[noeud] (a) at (  -2,   -3) {a};
			\node[noeud] (b) at (   2,   -3) {b};
			\node[noeud] (c) at (   0, -5.5) {c};
			\node[noeud] (d) at (  -2,   -8) {d};
			\node[noeud] (e) at (   2,   -8) {e};
			\node[noeud] (t) at (   0,  -11) {t};

			\draw[fleche] (s) to                  node [above  left] {10} (a);
			\draw[fleche] (s) to                  node [above right] { 1} (b);
			\draw[fleche] (a) to                  node [left       ] { 6} (d);
			\draw[fleche] (b) to                  node [right      ] { 5} (e);
			\draw[fleche] (c) to                  node [above right] { 2} (e);
			\draw[fleche] (d) to                  node [below  left] { 2} (t);
			\draw[fleche] (e) to                  node [below right] { 2} (t);
		\end{tikzpicture}
	\end{minipage}\hfill 
	\begin{minipage}[c]{0.3\linewidth}
		\begin{tabular}{|c|c|c|}
			\hline
			$i$ & $e(i)$ & $d(i)$ \\ \hline
			s & $\infty$ & 7* \\ \hline
			a &  0 & 2 \\ \hline
			b &  0 & 2 \\ \hline
			c &  0 & 2 \\ \hline
			d &  0 & 1 \\ \hline
			e &  0 & 1 \\ \hline
		\end{tabular}
	\end{minipage}
\end{frame}

% Poussage depuis s
\begin{frame}{Initialisation}
	\begin{minipage}[c]{0.3\linewidth}
		\begin{tikzpicture}[scale=0.55]
			\tikzset{noeud/.style={circle, draw=black, inner sep=0.1cm, minimum width=0.6cm}, fleche/.style={>=latex, ->}};

			\node[noeud] (s) at (   0,    0) {s};
			\node[noeud] (a) at (  -2,   -3) {a};
			\node[noeud] (b) at (   2,   -3) {b};
			\node[noeud] (c) at (   0, -5.5) {c};
			\node[noeud] (d) at (  -2,   -8) {d};
			\node[noeud] (e) at (   2,   -8) {e};
			\node[noeud] (t) at (   0,  -11) {t};

			\draw[fleche, red] (s) to                  node [above  left, red] {10} (a);
			\draw[fleche, red] (s) to                  node [above right, red] { 1} (b);
			\draw[fleche] (a) to                  node [left       ] {0} (d);
			\draw[fleche] (a) to[out=300, in=170] node [below  left] {0} (c);
			\draw[fleche] (c) to[out=120, in=345] node [above right] {0} (a);
			\draw[fleche] (c) to                  node [above  left] {0} (b);
			\draw[fleche] (b) to                  node [right      ] {0} (e);
			\draw[fleche] (c) to                  node [above right] {0} (e);
			\draw[fleche] (d) to[out= 15, in=165] node [above      ] {0} (e);
			\draw[fleche] (e) to[out=195, in=345] node [below      ] {0} (d);
			\draw[fleche] (d) to                  node [below  left] {0} (t);
			\draw[fleche] (e) to                  node [below right] {0} (t);
		\end{tikzpicture}
	\end{minipage}\hfill
	\begin{minipage}[c]{0.3\linewidth}
		\begin{tikzpicture}[scale=0.55]
			\tikzset{noeud/.style={circle, draw=black, inner sep=0.1cm, minimum width=0.6cm}, fleche/.style={>=latex, ->}};

			\node[noeud] (s) at (   0,    0) {s};
			\node[noeud] (a) at (  -2,   -3) {a};
			\node[noeud] (b) at (   2,   -3) {b};
			\node[noeud] (c) at (   0, -5.5) {c};
			\node[noeud] (d) at (  -2,   -8) {d};
			\node[noeud] (e) at (   2,   -8) {e};
			\node[noeud] (t) at (   0,  -11) {t};

			\draw[fleche] (a) to                  node [left       ] { 6} (d);
			\draw[fleche] (b) to                  node [right      ] { 5} (e);
			\draw[fleche] (c) to                  node [above right] { 2} (e);
			\draw[fleche] (d) to                  node [below  left] { 2} (t);
			\draw[fleche] (e) to                  node [below right] { 2} (t);
		\end{tikzpicture}
	\end{minipage}\hfill 
	\begin{minipage}[c]{0.3\linewidth}
		\begin{tabular}{|c|c|c|}
			\hline
			$i$ & $e(i)$ & $d(i)$ \\ \hline
			s & $\infty$ & 7* \\ \hline
			a & 10 & 2 \\ \hline
			b &  1 & 2 \\ \hline
			c &  0 & 2 \\ \hline
			d &  0 & 1 \\ \hline
			e &  0 & 1 \\ \hline
		\end{tabular}
	\end{minipage}
\end{frame}

\begin{frame}{Phase 1}
	\begin{minipage}[c]{0.3\linewidth}
		\begin{tikzpicture}[scale=0.55]
			\tikzset{noeud/.style={circle, draw=black, inner sep=0.1cm, minimum width=0.6cm}, fleche/.style={>=latex, ->}};

			\node[noeud] (s) at (   0,    0) {s};
			\node[noeud] (a) at (  -2,   -3) {a};
			\node[noeud] (b) at (   2,   -3) {b};
			\node[noeud] (c) at (   0, -5.5) {c};
			\node[noeud] (d) at (  -2,   -8) {d};
			\node[noeud] (e) at (   2,   -8) {e};
			\node[noeud] (t) at (   0,  -11) {t};

			\draw[fleche, red] (s) to                  node [above  left, red] {10} (a);
			\draw[fleche, red] (s) to                  node [above right, red] { 1} (b);
			\draw[fleche, red] (a) to                  node [left       , red] { 6} (d);
			\draw[fleche] (a) to[out=300, in=170] node [below  left] {0} (c);
			\draw[fleche] (c) to[out=120, in=345] node [above right] {0} (a);
			\draw[fleche] (c) to                  node [above  left] {0} (b);
			\draw[fleche, red] (b) to                  node [right      , red] { 1} (e);
			\draw[fleche] (c) to                  node [above right] {0} (e);
			\draw[fleche] (d) to[out= 15, in=165] node [above      ] {0} (e);
			\draw[fleche] (e) to[out=195, in=345] node [below      ] {0} (d);
			\draw[fleche] (d) to                  node [below  left] {0} (t);
			\draw[fleche] (e) to                  node [below right] {0} (t);
		\end{tikzpicture}
	\end{minipage}\hfill
	\begin{minipage}[c]{0.3\linewidth}
		\begin{tikzpicture}[scale=0.55]
			\tikzset{noeud/.style={circle, draw=black, inner sep=0.1cm, minimum width=0.6cm}, fleche/.style={>=latex, ->}};

			\node[noeud] (s) at (   0,    0) {s};
			\node[noeud] (a) at (  -2,   -3) {a};
			\node[noeud] (b) at (   2,   -3) {b};
			\node[noeud] (c) at (   0, -5.5) {c};
			\node[noeud] (d) at (  -2,   -8) {d};
			\node[noeud] (e) at (   2,   -8) {e};
			\node[noeud] (t) at (   0,  -11) {t};

			\draw[fleche] (b) to node [left       ] { 4} (e);
			\draw[fleche] (c) to                  node [above right] { 2} (e);
			\draw[fleche] (d) to                  node [below  left] { 2} (t);
			\draw[fleche] (e) to                  node [below right] { 2} (t);
		\end{tikzpicture}
	\end{minipage}\hfill 
	\begin{minipage}[c]{0.3\linewidth}
		\begin{tabular}{|c|c|c|}
			\hline
			$i$ & $e(i)$ & $d(i)$ \\ \hline
			s & $\infty$ & 7* \\ \hline
			a &  4 & 2 \\ \hline
			b &  0 & 2 \\ \hline
			c &  0 & 2 \\ \hline
			d &  6 & 1 \\ \hline
			e &  1 & 1 \\ \hline
		\end{tabular}
	\end{minipage}
\end{frame}

\begin{frame}{Phase 1}
	\begin{minipage}[c]{0.3\linewidth}
		\begin{tikzpicture}[scale=0.55]
			\tikzset{noeud/.style={circle, draw=black, inner sep=0.1cm, minimum width=0.6cm}, fleche/.style={>=latex, ->}};

			\node[noeud] (s) at (   0,    0) {s};
			\node[noeud] (a) at (  -2,   -3) {a};
			\node[noeud] (b) at (   2,   -3) {b};
			\node[noeud] (c) at (   0, -5.5) {c};
			\node[noeud] (d) at (  -2,   -8) {d};
			\node[noeud] (e) at (   2,   -8) {e};
			\node[noeud] (t) at (   0,  -11) {t};

			\draw[fleche, red] (s) to                  node [above  left, red] {10} (a);
			\draw[fleche, red] (s) to                  node [above right, red] { 1} (b);
			\draw[fleche, red] (a) to                  node [left       , red] { 6} (d);
			\draw[fleche] (a) to[out=300, in=170] node [below  left] {0} (c);
			\draw[fleche] (c) to[out=120, in=345] node [above right] {0} (a);
			\draw[fleche] (c) to                  node [above  left] {0} (b);
			\draw[fleche, red] (b) to                  node [right      , red] { 1} (e);
			\draw[fleche] (c) to                  node [above right] {0} (e);
			\draw[fleche] (d) to[out= 15, in=165] node [above      ] {0} (e);
			\draw[fleche] (e) to[out=195, in=345] node [below      ] {0} (d);
			\draw[fleche, red] (d) to                  node [below  left, red] { 2} (t);
			\draw[fleche, red] (e) to                  node [below right, red] { 1} (t);
		\end{tikzpicture}
	\end{minipage}\hfill
	\begin{minipage}[c]{0.3\linewidth}
		\begin{tikzpicture}[scale=0.55]
			\tikzset{noeud/.style={circle, draw=black, inner sep=0.1cm, minimum width=0.6cm}, fleche/.style={>=latex, ->}};

			\node[noeud] (s) at (   0,    0) {s};
			\node[noeud] (a) at (  -2,   -3) {a};
			\node[noeud] (b) at (   2,   -3) {b};
			\node[noeud] (c) at (   0, -5.5) {c};
			\node[noeud] (d) at (  -2,   -8) {d};
			\node[noeud] (e) at (   2,   -8) {e};
			\node[noeud] (t) at (   0,  -11) {t};

			\draw[fleche] (b) to node [left       ] { 4} (e);
			\draw[fleche] (c) to                  node [above right] { 2} (e);
			\draw[fleche] (e) to node [above  left] { 1} (t);
		\end{tikzpicture}
	\end{minipage}\hfill 
	\begin{minipage}[c]{0.3\linewidth}
		\begin{tabular}{|c|c|c|}
			\hline
			$i$ & $e(i)$ & $d(i)$ \\ \hline
			s & $\infty$ & 7* \\ \hline
			a &  4 & 2 \\ \hline
			b &  0 & 2 \\ \hline
			c &  0 & 2 \\ \hline
			d &  4 & 1 \\ \hline
			e &  0 & 1 \\ \hline
		\end{tabular}
	\end{minipage}
\end{frame}

\begin{frame}{Phase 2}
	\begin{minipage}[c]{0.3\linewidth}
		\begin{tikzpicture}[scale=0.55]
			\tikzset{noeud/.style={circle, draw=black, inner sep=0.1cm, minimum width=0.6cm}, fleche/.style={>=latex, ->}};

			\node[noeud] (s) at (   0,    0) {s};
			\node[noeud] (a) at (  -2,   -3) {a};
			\node[noeud] (b) at (   2,   -3) {b};
			\node[noeud] (c) at (   0, -5.5) {c};
			\node[noeud] (d) at (  -2,   -8) {d};
			\node[noeud] (e) at (   2,   -8) {e};
			\node[noeud] (t) at (   0,  -11) {t};

			\draw[fleche, red] (s) to                  node [above  left, red] {10} (a);
			\draw[fleche, red] (s) to                  node [above right, red] { 1} (b);
			\draw[fleche, red] (a) to                  node [left       , red] { 6} (d);
			\draw[fleche] (a) to[out=300, in=170] node [below  left] {0} (c);
			\draw[fleche] (c) to[out=120, in=345] node [above right] {0} (a);
			\draw[fleche] (c) to                  node [above  left] {0} (b);
			\draw[fleche, red] (b) to                  node [right      , red] { 1} (e);
			\draw[fleche] (c) to                  node [above right] {0} (e);
			\draw[fleche] (d) to[out= 15, in=165] node [above      ] {0} (e);
			\draw[fleche] (e) to[out=195, in=345] node [below      ] {0} (d);
			\draw[fleche, red] (d) to                  node [below  left, red] { 2} (t);
			\draw[fleche, red] (e) to                  node [below right, red] { 1} (t);
		\end{tikzpicture}
	\end{minipage}\hfill
	\begin{minipage}[c]{0.3\linewidth}
		\begin{tikzpicture}[scale=0.55]
			\tikzset{noeud/.style={circle, draw=black, inner sep=0.1cm, minimum width=0.6cm}, fleche/.style={>=latex, ->}};

			\node[noeud] (s) at (   0,    0) {s};
			\node[noeud] (a) at (  -2,   -3) {a};
			\node[noeud] (b) at (   2,   -3) {b};
			\node[noeud] (c) at (   0, -5.5) {c};
			\node[noeud] (d) at (  -2,   -8) {d};
			\node[noeud] (e) at (   2,   -8) {e};
			\node[noeud] (t) at (   0,  -11) {t};

			\draw[fleche] (b) to node [left       ] { 4} (e);
			\draw[fleche] (c) to node [above right] { 2} (e);
			\draw[fleche] (d) to node [above      ] { 3} (e);
			\draw[fleche] (e) to node [above  left] { 1} (t);
		\end{tikzpicture}
	\end{minipage}\hfill 
	\begin{minipage}[c]{0.3\linewidth}
		\begin{tabular}{|c|c|c|}
			\hline
			$i$ & $e(i)$ & $d(i)$ \\ \hline
			s & $\infty$ & 7* \\ \hline
			a &  4 & 2 \\ \hline
			b &  0 & 2 \\ \hline
			c &  0 & 2 \\ \hline
			\textcolor{red}{d} &  \textcolor{red}{4} & \textcolor{red}{2} \\ \hline
			e &  0 & 1 \\ \hline
		\end{tabular}
	\end{minipage}
\end{frame}

\begin{frame}{Phase 2}
	\begin{minipage}[c]{0.3\linewidth}
		\begin{tikzpicture}[scale=0.55]
			\tikzset{noeud/.style={circle, draw=black, inner sep=0.1cm, minimum width=0.6cm}, fleche/.style={>=latex, ->}};

			\node[noeud] (s) at (   0,    0) {s};
			\node[noeud] (a) at (  -2,   -3) {a};
			\node[noeud] (b) at (   2,   -3) {b};
			\node[noeud] (c) at (   0, -5.5) {c};
			\node[noeud] (d) at (  -2,   -8) {d};
			\node[noeud] (e) at (   2,   -8) {e};
			\node[noeud] (t) at (   0,  -11) {t};

			\draw[fleche, red] (s) to                  node [above  left, red] {10} (a);
			\draw[fleche, red] (s) to                  node [above right, red] { 1} (b);
			\draw[fleche, red] (a) to                  node [left       , red] { 6} (d);
			\draw[fleche] (a) to[out=300, in=170] node [below  left] {0} (c);
			\draw[fleche] (c) to[out=120, in=345] node [above right] {0} (a);
			\draw[fleche] (c) to                  node [above  left] {0} (b);
			\draw[fleche, red] (b) to                  node [right      , red] { 1} (e);
			\draw[fleche] (c) to                  node [above right] {0} (e);
			\draw[fleche, red] (d) to[out= 15, in=165] node [above      , red] { 3} (e);
			\draw[fleche] (e) to[out=195, in=345] node [below      ] {0} (d);
			\draw[fleche, red] (d) to                  node [below  left, red] { 2} (t);
			\draw[fleche, red] (e) to                  node [below right, red] { 1} (t);
		\end{tikzpicture}
	\end{minipage}\hfill
	\begin{minipage}[c]{0.3\linewidth}
		\begin{tikzpicture}[scale=0.55]
			\tikzset{noeud/.style={circle, draw=black, inner sep=0.1cm, minimum width=0.6cm}, fleche/.style={>=latex, ->}};

			\node[noeud] (s) at (   0,    0) {s};
			\node[noeud] (a) at (  -2,   -3) {a};
			\node[noeud] (b) at (   2,   -3) {b};
			\node[noeud] (c) at (   0, -5.5) {c};
			\node[noeud] (d) at (  -2,   -8) {d};
			\node[noeud] (e) at (   2,   -8) {e};
			\node[noeud] (t) at (   0,  -11) {t};

			\draw[fleche] (b) to node [left       ] { 4} (e);
			\draw[fleche] (c) to node [above right] { 2} (e);
			\draw[fleche] (e) to node [above  left] { 1} (t);
		\end{tikzpicture}
	\end{minipage}\hfill 
	\begin{minipage}[c]{0.3\linewidth}
		\begin{tabular}{|c|c|c|}
			\hline
			$i$ & $e(i)$ & $d(i)$ \\ \hline
			s & $\infty$ & 7 \\ \hline
			a &  4 & 2 \\ \hline
			b &  0 & 2 \\ \hline
			c &  0 & 2 \\ \hline
			d &  1 & 2 \\ \hline
			e &  3 & 1 \\ \hline
		\end{tabular}
	\end{minipage}
\end{frame}

\begin{frame}{Phase 2}
	\begin{minipage}[c]{0.3\linewidth}
		\begin{tikzpicture}[scale=0.55]
			\tikzset{noeud/.style={circle, draw=black, inner sep=0.1cm, minimum width=0.6cm}, fleche/.style={>=latex, ->}};

			\node[noeud] (s) at (   0,    0) {s};
			\node[noeud] (a) at (  -2,   -3) {a};
			\node[noeud] (b) at (   2,   -3) {b};
			\node[noeud] (c) at (   0, -5.5) {c};
			\node[noeud] (d) at (  -2,   -8) {d};
			\node[noeud] (e) at (   2,   -8) {e};
			\node[noeud] (t) at (   0,  -11) {t};

			\draw[fleche, red] (s) to                  node [above  left, red] {10} (a);
			\draw[fleche, red] (s) to                  node [above right, red] { 1} (b);
			\draw[fleche, red] (a) to                  node [left       , red] { 6} (d);
			\draw[fleche] (a) to[out=300, in=170] node [below  left] {0} (c);
			\draw[fleche] (c) to[out=120, in=345] node [above right] {0} (a);
			\draw[fleche] (c) to                  node [above  left] {0} (b);
			\draw[fleche, red] (b) to                  node [right      , red] { 1} (e);
			\draw[fleche] (c) to                  node [above right] {0} (e);
			\draw[fleche, red] (d) to[out= 15, in=165] node [above      , red] { 3} (e);
			\draw[fleche] (e) to[out=195, in=345] node [below      ] {0} (d);
			\draw[fleche, red] (d) to                  node [below  left, red] { 2} (t);
			\draw[fleche, red] (e) to                  node [below right, red] { 2} (t);
		\end{tikzpicture}
	\end{minipage}\hfill
	\begin{minipage}[c]{0.3\linewidth}
		\begin{tikzpicture}[scale=0.55]
			\tikzset{noeud/.style={circle, draw=black, inner sep=0.1cm, minimum width=0.6cm}, fleche/.style={>=latex, ->}};

			\node[noeud] (s) at (   0,    0) {s};
			\node[noeud] (a) at (  -2,   -3) {a};
			\node[noeud] (b) at (   2,   -3) {b};
			\node[noeud] (c) at (   0, -5.5) {c};
			\node[noeud] (d) at (  -2,   -8) {d};
			\node[noeud] (e) at (   2,   -8) {e};
			\node[noeud] (t) at (   0,  -11) {t};

			\draw[fleche] (b) to[out=255, in=105] node [left       ] { 4} (e);
			\draw[fleche] (c) to                  node [above right] { 2} (e);
		\end{tikzpicture}
	\end{minipage}\hfill 
	\begin{minipage}[c]{0.3\linewidth}
		\begin{tabular}{|c|c|c|}
			\hline
			$i$ & $e(i)$ & $d(i)$ \\ \hline
			s & $\infty$ & 7 \\ \hline
			a &  4 & 2 \\ \hline
			b &  0 & 2 \\ \hline
			c &  0 & 2 \\ \hline
			d &  1 & 2 \\ \hline
			e &  2 & 1 \\ \hline
		\end{tabular}
	\end{minipage}
\end{frame}

\begin{frame}{Phase 3}
	\begin{minipage}[c]{0.3\linewidth}
		\begin{tikzpicture}[scale=0.55]
			\tikzset{noeud/.style={circle, draw=black, inner sep=0.1cm, minimum width=0.6cm}, fleche/.style={>=latex, ->}};

			\node[noeud] (s) at (   0,    0) {s};
			\node[noeud] (a) at (  -2,   -3) {a};
			\node[noeud] (b) at (   2,   -3) {b};
			\node[noeud] (c) at (   0, -5.5) {c};
			\node[noeud] (d) at (  -2,   -8) {d};
			\node[noeud] (e) at (   2,   -8) {e};
			\node[noeud] (t) at (   0,  -11) {t};

			\draw[fleche, red] (s) to                  node [above  left, red] {10} (a);
			\draw[fleche, red] (s) to                  node [above right, red] { 1} (b);
			\draw[fleche, red] (a) to                  node [left       , red] { 6} (d);
			\draw[fleche] (a) to[out=300, in=170] node [below  left] {0} (c);
			\draw[fleche] (c) to[out=120, in=345] node [above right] {0} (a);
			\draw[fleche] (c) to                  node [above  left] {0} (b);
			\draw[fleche, red] (b) to                  node [right      , red] { 1} (e);
			\draw[fleche] (c) to                  node [above right] {0} (e);
			\draw[fleche, red] (d) to[out= 15, in=165] node [above      , red] { 3} (e);
			\draw[fleche] (e) to[out=195, in=345] node [below      ] {0} (d);
			\draw[fleche, red] (d) to                  node [below  left, red] { 2} (t);
			\draw[fleche, red] (e) to                  node [below right, red] { 2} (t);
		\end{tikzpicture}
	\end{minipage}\hfill
	\begin{minipage}[c]{0.3\linewidth}
		\begin{tikzpicture}[scale=0.55]
			\tikzset{noeud/.style={circle, draw=black, inner sep=0.1cm, minimum width=0.6cm}, fleche/.style={>=latex, ->}};

			\node[noeud] (s) at (   0,    0) {s};
			\node[noeud] (a) at (  -2,   -3) {a};
			\node[noeud] (b) at (   2,   -3) {b};
			\node[noeud] (c) at (   0, -5.5) {c};
			\node[noeud] (d) at (  -2,   -8) {d};
			\node[noeud] (e) at (   2,   -8) {e};
			\node[noeud] (t) at (   0,  -11) {t};

			\draw[fleche, green] (e) to node [right      , green] { 1} (b);
			\draw[fleche, green] (e) to node [below      , green] { 6} (d);
		\end{tikzpicture}
	\end{minipage}\hfill 
	\begin{minipage}[c]{0.3\linewidth}
		\begin{tabular}{|c|c|c|}
			\hline
			$i$ & $e(i)$ & $d(i)$ \\ \hline
			s & $\infty$ & 7* \\ \hline
			a &  4 & 2 \\ \hline
			b &  0 & 2 \\ \hline
			c &  0 & 2 \\ \hline
			d &  1 & 2 \\ \hline
			\textcolor{red}{e} &  \textcolor{red}{2} & \textcolor{red}{3} \\ \hline
		\end{tabular}
	\end{minipage}
\end{frame}

\begin{frame}{Phase 3}
	\begin{minipage}[c]{0.3\linewidth}
		\begin{tikzpicture}[scale=0.55]
			\tikzset{noeud/.style={circle, draw=black, inner sep=0.1cm, minimum width=0.6cm}, fleche/.style={>=latex, ->}};

			\node[noeud] (s) at (   0,    0) {s};
			\node[noeud] (a) at (  -2,   -3) {a};
			\node[noeud] (b) at (   2,   -3) {b};
			\node[noeud] (c) at (   0, -5.5) {c};
			\node[noeud] (d) at (  -2,   -8) {d};
			\node[noeud] (e) at (   2,   -8) {e};
			\node[noeud] (t) at (   0,  -11) {t};

			\draw[fleche, red] (s) to                  node [above  left, red] {10} (a);
			\draw[fleche, red] (s) to                  node [above right, red] { 1} (b);
			\draw[fleche, red] (a) to                  node [left       , red] { 6} (d);
			\draw[fleche] (a) to[out=300, in=170] node [below  left] {0} (c);
			\draw[fleche] (c) to[out=120, in=345] node [above right] {0} (a);
			\draw[fleche] (c) to                  node [above  left] {0} (b);
			\draw[fleche, red] (b) to                  node [right      , red] { 1} (e);
			\draw[fleche] (c) to                  node [above right] {0} (e);
			\draw[fleche, red] (d) to[out= 15, in=165] node [above      , red] { 1} (e);
			\draw[fleche] (e) to[out=195, in=345] node [below      ] {0} (d);
			\draw[fleche, red] (d) to                  node [below  left, red] { 2} (t);
			\draw[fleche, red] (e) to                  node [below right, red] { 2} (t);
		\end{tikzpicture}
	\end{minipage}\hfill
	\begin{minipage}[c]{0.3\linewidth}
		\begin{tikzpicture}[scale=0.55]
			\tikzset{noeud/.style={circle, draw=black, inner sep=0.1cm, minimum width=0.6cm}, fleche/.style={>=latex, ->}};

			\node[noeud] (s) at (   0,    0) {s};
			\node[noeud] (a) at (  -2,   -3) {a};
			\node[noeud] (b) at (   2,   -3) {b};
			\node[noeud] (c) at (   0, -5.5) {c};
			\node[noeud] (d) at (  -2,   -8) {d};
			\node[noeud] (e) at (   2,   -8) {e};
			\node[noeud] (t) at (   0,  -11) {t};

			\draw[fleche, green] (e) to node [right      , green] { 1} (b);
			\draw[fleche, green] (e) to node [below      , green] {4} (d);
		\end{tikzpicture}
	\end{minipage}\hfill 
	\begin{minipage}[c]{0.3\linewidth}
		\begin{tabular}{|c|c|c|}
			\hline
			$i$ & $e(i)$ & $d(i)$ \\ \hline
			s & $\infty$ & 7 \\ \hline
			a &  4 & 2 \\ \hline
			b &  0 & 2 \\ \hline
			c &  0 & 2 \\ \hline
			d &  3 & 2 \\ \hline
			e &  0 & 3 \\ \hline
		\end{tabular}
	\end{minipage}
\end{frame}

\begin{frame}{Phase 4}
	\begin{minipage}[c]{0.3\linewidth}
		\begin{tikzpicture}[scale=0.55]
			\tikzset{noeud/.style={circle, draw=black, inner sep=0.1cm, minimum width=0.6cm}, fleche/.style={>=latex, ->}};

			\node[noeud] (s) at (   0,    0) {s};
			\node[noeud] (a) at (  -2,   -3) {a};
			\node[noeud] (b) at (   2,   -3) {b};
			\node[noeud] (c) at (   0, -5.5) {c};
			\node[noeud] (d) at (  -2,   -8) {d};
			\node[noeud] (e) at (   2,   -8) {e};
			\node[noeud] (t) at (   0,  -11) {t};

			\draw[fleche, red] (s) to                  node [above  left, red] {10} (a);
			\draw[fleche, red] (s) to                  node [above right, red] { 1} (b);
			\draw[fleche, red] (a) to                  node [left       , red] { 6} (d);
			\draw[fleche] (a) to[out=300, in=170] node [below  left] {0} (c);
			\draw[fleche] (c) to[out=120, in=345] node [above right] {0} (a);
			\draw[fleche] (c) to                  node [above  left] {0} (b);
			\draw[fleche, red] (b) to                  node [right      , red] { 1} (e);
			\draw[fleche] (c) to                  node [above right] {0} (e);
			\draw[fleche, red] (d) to[out= 15, in=165] node [above      , red] { 1} (e);
			\draw[fleche] (e) to[out=195, in=345] node [below      ] {0} (d);
			\draw[fleche, red] (d) to                  node [below  left, red] { 2} (t);
			\draw[fleche, red] (e) to                  node [below right, red] { 2} (t);
		\end{tikzpicture}
	\end{minipage}\hfill
	\begin{minipage}[c]{0.3\linewidth}
		\begin{tikzpicture}[scale=0.55]
			\tikzset{noeud/.style={circle, draw=black, inner sep=0.1cm, minimum width=0.6cm}, fleche/.style={>=latex, ->}};

			\node[noeud] (s) at (   0,    0) {s};
			\node[noeud] (a) at (  -2,   -3) {a};
			\node[noeud] (b) at (   2,   -3) {b};
			\node[noeud] (c) at (   0, -5.5) {c};
			\node[noeud] (d) at (  -2,   -8) {d};
			\node[noeud] (e) at (   2,   -8) {e};
			\node[noeud] (t) at (   0,  -11) {t};

			\draw[fleche, green] (d) to                  node [left       , green] { 6} (a);
			\draw[fleche, green] (e) to node [right      , green] { 1} (b);
		\end{tikzpicture}
	\end{minipage}\hfill 
	\begin{minipage}[c]{0.3\linewidth}
		\begin{tabular}{|c|c|c|}
			\hline
			$i$ & $e(i)$ & $d(i)$ \\ \hline
			s & $\infty$ & 7* \\ \hline
			a &  4 & 2 \\ \hline
			b &  0 & 2 \\ \hline
			c &  0 & 2 \\ \hline
			\textcolor{red}{d} &  \textcolor{red}{3} & \textcolor{red}{3} \\ \hline
			e &  0 & 3 \\ \hline
		\end{tabular}
	\end{minipage}
\end{frame}

\begin{frame}{Quelques tours d'algorithme plus tard}
	\begin{minipage}[c]{0.3\linewidth}
		\begin{tikzpicture}[scale=0.55]
			\tikzset{noeud/.style={circle, draw=black, inner sep=0.1cm, minimum width=0.6cm}, fleche/.style={>=latex, ->}};

			\node[noeud] (s) at (   0,    0) {s};
			\node[noeud] (a) at (  -2,   -3) {a};
			\node[noeud] (b) at (   2,   -3) {b};
			\node[noeud] (c) at (   0, -5.5) {c};
			\node[noeud] (d) at (  -2,   -8) {d};
			\node[noeud] (e) at (   2,   -8) {e};
			\node[noeud] (t) at (   0,  -11) {t};

			\draw[fleche, red] (s) to                  node [above  left, red] { 3} (a);
			\draw[fleche, red] (s) to                  node [above right, red] { 1} (b);
			\draw[fleche, red] (a) to                  node [left       , red] { 3} (d);
			\draw[fleche, red] (b) to                  node [right      , red] { 1} (e);
			\draw[fleche, red] (d) to                  node [above      , red] { 1} (e);
			\draw[fleche, red] (d) to                  node [below  left, red] { 2} (t);
			\draw[fleche, red] (e) to                  node [below right, red] { 2} (t);
		\end{tikzpicture}
	\end{minipage}\hfill
	\begin{minipage}[c]{0.3\linewidth}
		\begin{tikzpicture}[scale=0.55]
			\tikzset{noeud/.style={circle, draw=black, inner sep=0.1cm, minimum width=0.6cm}, fleche/.style={>=latex, ->}};

			\node[noeud] (s) at (   0,    0) {s};
			\node[noeud] (a) at (  -2,   -3) {a};
			\node[noeud] (b) at (   2,   -3) {b};
			\node[noeud] (c) at (   0, -5.5) {c};
			\node[noeud] (d) at (  -2,   -8) {d};
			\node[noeud] (e) at (   2,   -8) {e};
			\node[noeud] (t) at (   0,  -11) {t};

			\draw[fleche, green] (a) to node [above  left, green] {3} (s);
			\draw[fleche, green] (b) to                  node [above right, green] { 1} (s);
		\end{tikzpicture}
	\end{minipage}\hfill 
	\begin{minipage}[c]{0.3\linewidth}
		\begin{tabular}{|c|c|c|}
			\hline
			$i$ & $e(i)$ & $d(i)$ \\ \hline
			s & $\infty$ & 7 \\ \hline
			a &  0 & 8 \\ \hline
			b &  0 & 8 \\ \hline
			c &  0 & 8 \\ \hline
			d &  0 & 8 \\ \hline
			e &  0 & 8 \\ \hline
		\end{tabular}
	\end{minipage}
\end{frame}

\subsection{Les algorithmes dérivés}

\begin{frame}{Une question de choix}
	Les algorithmes dérivés sont basés sur un choix plus judicieux des sommets actfis à traiter.
	\vfill
\end{frame}

\begin{frame}{Une question de choix}
	Les algorithmes dérivés sont basés sur un choix plus judicieux des sommets actfis à traiter.
	\vfill
	\textbf{L'algorithme FIFO :} dès qu'un noeud devient actif, on le place dans une file. On traite
	alors les sommets de la file dans l'ordre \emph{First In First Out}\vfill
\end{frame}

\begin{frame}{Une question de choix}
	Les algorithmes dérivés sont basés sur un choix plus judicieux des sommets actfis à traiter.
	\vfill
	\textbf{L'algorithme FIFO :} dès qu'un noeud devient actif, on le place dans une file. On traite
	alors les sommets de la file dans l'ordre \emph{First In First Out}\vfill
	\textbf{L'algorithme High Label :} on choisit de préférence, le noeud actif ayant la plus grande
	distance au puits \vfill
\end{frame}

\begin{frame}{Complexités}
	\renewcommand{\arraystretch}{2.5}
	\begin{tabular}{c|p {0.12\linewidth}|p {0.35\linewidth}|p {0.35\linewidth}|} \cline{3-4}
		\multicolumn{2}{c|}{}& Algorithme & Complexité \\ \cline{2-4}
		&\multirow{2}{*}{$\quad$\rotatebox{90}{\parbox[b]{5.1em}{\centering Chaînes\\ améliorantes~}}} & Edmonds-Karp & $O(SA^2)$ \\
		\cline{3-4}
	\renewcommand{\arraystretch}{2}
		& & Dinic & $O(S^2A)$ \\\cline{2-4}
		&\multirow{3}{*}{$\quad$\rotatebox{90}{\parbox[t]{5.5em}{\centering Poussage\\ Réétiquetage}}} &
		Générique & $O(S^2A)$ \\ \cline{3-4}
		& & FIFO & $O(S^3)$ \\ \cline{3-4}
		& & High Label & $O(S^2\sqrt{A})$ \\ \cline{2-4}
	\end{tabular}
\end{frame}

\section{Les tests}

\subsection{Le programme}

\begin{frame}{Générateur de graphe aléatoire}
	\begin{itemize}
		\item Deux états "relié" et "non relié"
	\end{itemize}
\end{frame}

\begin{frame}{Générateur de graphe aléatoire}
	\begin{itemize}
		\item Deux états "relié" et "non relié"
		\item Initialise tous les noeuds sauf la source à "non relié"
	\end{itemize}
\end{frame}

\begin{frame}{Générateur de graphe aléatoire}
	\begin{itemize}
		\item Deux états "relié" et "non relié"
		\item Initialise tous les noeuds sauf la source à "non relié"
		\item Construit un arbre
	\end{itemize}
\end{frame}

\begin{frame}{Générateur de graphe aléatoire}
	\begin{itemize}
		\item Deux états "relié" et "non relié"
		\item Initialise tous les noeuds sauf la source à "non relié"
		\item Construit un arbre
		\item Calcule le nombre de voisins (loi de Poisson)
	\end{itemize}
\end{frame}

\begin{frame}{Générateur de graphe aléatoire}
	\begin{itemize}
		\item Deux états "relié" et "non relié"
		\item Initialise tous les noeuds sauf la source à "non relié"
		\item Construit un arbre
		\item Calcule le nombre de voisins (loi de Poisson)
		\item Relie les points aléatoirement
	\end{itemize}
\end{frame}

\subsection{Modus Operandi}

\begin{frame}{Les tests}
	\begin{itemize}
		\item Définit un ratio : $r = \frac{m}{n}$
	\end{itemize}
\end{frame}

\begin{frame}{Les tests}
	\begin{itemize}
		\item Définit un ratio : $r = \frac{m}{n}$
		\item Fait varier le nombre de noeuds
	\end{itemize}
\end{frame}
\begin{frame}{Les tests}
	\begin{itemize}
		\item Définit un ratio : $r = \frac{m}{n}$
		\item Fait varier le nombre de noeuds
		\item Réalise les trois algorithmes sur 100 graphes aléatoires différents
	\end{itemize}
\end{frame}
\begin{frame}{Les tests}
	\begin{itemize}
		\item Définit un ratio : $r = \frac{m}{n}$
		\item Fait varier le nombre de noeuds
		\item Réalise les trois algorithmes sur 100 graphes aléatoires différents
		\item Trace les courbes
	\end{itemize}
\end{frame}

\subsection{Retour d'expériences}

\begin{frame}{Les résultats attendus}
	Pour un ratio de $n$ : 
\end{frame}

\begin{frame}{Les résultats attendus}
	Pour un ratio de $n$ : \begin{itemize}
		\item Une complexité en $O(n^4)$ pour Dinic
	\end{itemize}
\end{frame}

\begin{frame}{Les résultats attendus}
	Pour un ratio de $n$ : \begin{itemize}
		\item Une complexité en $O(n^4)$ pour Dinic
		\item Une complexité en $O(n^3)$ pour les algorithmes FIFO et High Label
	\end{itemize}
\end{frame}

\begin{frame}{Les résultats attendus}
	Pour un ratio de $n$ : \begin{itemize}
		\item Une complexité en $O(n^4)$ pour Dinic
		\item Une complexité en $O(n^3)$ pour les algorithmes FIFO et High Label
		\item Une exécution plus rapide pour FIFO et High Label
	\end{itemize}
\end{frame}

\begin{frame}{Les résultats obtenus}
	\begin{minipage}[c]{0.50\linewidth}
		\includegraphics[scale=0.3]{img/resultat.png}
	\end{minipage}\hfill
	\begin{minipage}[c]{0.40\linewidth}
	\end{minipage}
\end{frame}

\begin{frame}{Les résultats obtenus}
	\begin{minipage}[c]{0.50\linewidth}
		\includegraphics[scale=0.3]{img/resultat.png}
	\end{minipage}\hfill
	\begin{minipage}[c]{0.40\linewidth}
		\begin{itemize}
			\item $O(n^4)$ pour tous les algorithmes
		\end{itemize}
	\end{minipage}
\end{frame}

\begin{frame}{Les résultats obtenus}
	\begin{minipage}[c]{0.50\linewidth}
		\includegraphics[scale=0.3]{img/resultat.png}
	\end{minipage}\hfill
	\begin{minipage}[c]{0.40\linewidth}
		\begin{itemize}
			\item $O(n^4)$ pour tous les algorithmes
			\item Dinic plus rapide
		\end{itemize}
	\end{minipage}
\end{frame}

\begin{frame}{Analyse des résultats}
	Les causes plausibles de ce résultats sont :
\end{frame}

\begin{frame}{Analyse des résultats}
	Les causes plausibles de ce résultats sont :
	\begin{itemize}
		\item Erreur d'appréhension du réseau résiduel
	\end{itemize}
\end{frame}

\begin{frame}{Analyse des résultats}
	Les causes plausibles de ce résultats sont :
	\begin{itemize}
		\item Erreur d'appréhension du réseau résiduel
		\item Erreur d'implémentation
	\end{itemize}
\end{frame}

\begin{frame}{Analyse des résultats}
	Les causes plausibles de ce résultats sont :
	\begin{itemize}
		\item Erreur d'appréhension du réseau résiduel
		\item Erreur d'implémentation
		\item Pré-traitement très long
	\end{itemize}
\end{frame}

\section{Conclusion}

\begin{frame}{Etat de l'art}
	\begin{itemize}
		\item Changement de conception des réseaux résiduels
	\end{itemize}
\end{frame}

\begin{frame}{Etat de l'art}
	\begin{itemize}
		\item Changement de conception des réseaux résiduels
		\item Tests de structures de graphe différentes (implémentation matricielles)
	\end{itemize}
\end{frame}

\begin{frame}{Etat de l'art}
	\begin{itemize}
		\item Changement de conception des réseaux résiduels
		\item Tests de structures de graphe différentes (implémentation matricielles)
		\item Recherche des différentes améliorations découvertes
	\end{itemize}
\end{frame}

\begin{frame}{Conclusion}
	\begin{itemize}
		\item Algorithmes intéressants (complexités, principes, fonctions de préflots, ...)
	\end{itemize}
\end{frame}

\begin{frame}{Conclusion}
	\begin{itemize}
		\item Algorithmes intéressants (complexités, principes, fonctions de préflots, ...)
		\item Parmis les plus rapides en pratique et en théorie
	\end{itemize}
\end{frame}

\begin{frame}{Conclusion}
	\begin{itemize}
		\item Algorithmes intéressants (complexités, principes, fonctions de préflots, ...)
		\item Parmis les plus rapides en pratique et en théorie
		\item Un travail du code est donc nécessaire
	\end{itemize}
\end{frame}

\end{document}
