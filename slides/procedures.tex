
\begin{frame}{Initialisation}
	Consiste à calculer les distances de chacun des noeuds de façon à obtenir une fonction de distance
	valide.\vfill
\end{frame}

\begin{frame}{Initialisation}
	Consiste à calculer les distances de chacun des noeuds de façon à obtenir une fonction de distance
	valide.\vfill
	Procédure d'initialisation : 
	\begin{algorithmic}[1]
			\FOR{$i\ \in\ S$}
				\STATE Calculer la distance $d(i)$ en nombre d'arêtes de $i$ à $t$  
			\ENDFOR
			\FOR{$a$ $\in$ $A(s)$}
				\STATE $x(a)\ \leftarrow$ $c(a)$  
			\ENDFOR
			\STATE $d(s)\ \leftarrow |S|$ 
	\end{algorithmic}
\end{frame}

\begin{frame}{Examen des noeuds}
	Consiste à sélectionner un noeud actif et à effectuer un poussage si possible, ou un ré-étiquetage
	sinon. \vfill
\end{frame}

\begin{frame}{Examen des noeuds}
	Consiste à sélectionner un noeud actif et à effectuer un poussage si possible, ou un ré-étiquetage
	sinon. \vfill

	Procédure Pousser-Réétiqueter :
	\begin{algorithmic}[1]
		\IF {$e(i) > 0$}
		\IF {$\exists j$ tel que $c(i,j) > 0$ \AND $d(i) = d(j) + 1$}
				\STATE $\delta \leftarrow \min(e(i), c(i,j))$
				\STATE $f(i,j) \leftarrow f(i,j) + \delta$
				\STATE $e(i) \leftarrow e(i) - \delta$
				\STATE $e(j) \leftarrow e(j) + \delta$
			\ELSE
				\STATE $d(i) \leftarrow 1 + \min\{ d(j) / (i,j) \in A_f\}$
			\ENDIF
		\ENDIF
	\end{algorithmic}\vfill
\end{frame}

\begin{frame}{La procédure principale}
	Consiste à initialiser le problème, et parcourir les noeuds pour l'application du poussage et
	ré-étiquetage.\vfill

	Algorithme Générique :
	\begin{algorithmic}[1]
			\STATE Initialisation() \\
			\WHILE{Il existe un noeud actif $i$}
				\STATE Pousser-Réétiqueter(i) 
			\ENDWHILE
	\end{algorithmic}\vfill
\end{frame}


