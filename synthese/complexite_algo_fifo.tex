Les améliorations sur l'algorithme générique se font soit grâce à l'amélioration de la structure de
donnée, soit à l'aide d'une amélioration sur la manière de sélectionner les noeuds actifs dans le
réseau de départ.

Définissons à présent la notion d'\emph{analyse de noeud}. Au cours de l'algorithme précédent, dans
le cas d'une opération de poussage saturant, nous avons vu que le noeud réalisant l'opération reste,
très probablement, un noeud actif, il est donc candidat à la sélection lors de la prochaine boucle
de l'algorithme. Seulement rien ne garantit qu'il sera à nouveau choisi, il peut donc être
intéressant dans ce genre de situation de réaliser plusieurs poussages saturants suivis soit d'un
poussage non saturant, soit d'une opération de réétiquetage. C'est cette suite d'opérations de que
nous appellerons dorénavant \emph{analyse de noeud}.

\begin{thrm}
	L'algorithme de préflot FIFO s'exécute en un temps $O(n^3)$.
\end{thrm}

\textbf{Preuve} \\
Afin de calculer la borne supérieure du temps d'exécution de l'algorithme, nous allons être	amenés à
introduire de le concept de phase. On appelle première phase, la partie de l'algorithme qui consiste
à appliquer une analyse de noeud à chacun des sommets devenus actifs suite aux opérations de
poussage effectuées sur $s$. Durant cette première phase, les noeuds devenus actifs suite aux
analyse de noeud des sommet de la première phase, sont mis en queue de file et leur analyse
constituera les opérations de la seconde phase, et ainsi de suite. L'objectif est donc de borner le
nombre de phases.

Considérons la fonction $\Phi = max \{d(i) / i \mbox{ actif }\}$. Considérons à présent l'effet
d'une phase sur la fonction $\Phi$, on remarque alors plusieurs cas :
\begin{enumerate}
	\item Lors de la phase aucune opération de réétiquetage n'a lieu. Dans ce cas, les sommets $i$ de
		cette phase ont tous effectués une opération de poussage non saturant, ils ne sont donc plus
		actifs. Les noeuds actifs après la phase $j$ ont tous une distance respectant cette loi : $d(j)
		< d(i) \quad \forall i, j$. $\Phi$ diminue alors d'au moins une unité.
	\item Lors de la phase il y a au moins une opération de ré étiquetage. $\Phi$ voit alors sa sa
		valeur augmenter de $\delta$ avec $\delta \geq 0$. Si $\delta = 0$, alors il existe un noeud qui
		voit sa hauteur augmenter au moins d'une unité, si $\delta > 0$ alors il existe un noeud dont la
		hauteur augment d'au moins $\delta$. De par le lemme \ref{borne_reetiq} le nombre de phases
		pendant lesquelles $\Phi$ est inférieure à $2n^2$.
\end{enumerate}
Comme la valeur initiale de $\Phi$ est $n$, le nombre de phases pendant lesquelles $\Phi$ diminue
est donc bornée supérieurement par : $2n^2 + n$, le nombre total de phases est donc borné par $4n^2
+ n$. De plus comme il y a au moins une opération de poussage non saturant par phase, le nombre de
poussage non saturants est borné par $4n^3 + n$. L'algorithme s'exécute donc en $O(n^3)$.
