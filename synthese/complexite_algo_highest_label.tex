Comme vu précédemment, l'algorithme High Label analyse les n\oe uds en fonction de la valeur de leur
fonction de distance. Cette sélection permet d'obtenir une borne maximale en $O(n^2\sqrt{m})$, c'est
ce que nous allons démontrer.

Soit une constante $K = \sqrt{m}$, soit la fonction $d': S \longrightarrow \mathbb{N}$, définie par
: $$
d'(i) = |\{j : d(j) \leq d(i) \}| \quad \forall i,j \in S $$
Autrement dit, $d'(i)$ représente le nombre de n\oe uds de $S$ étant plus proche du puits que $i$.
Cette fonctions vérifie alors quelques propriétés : 
\begin{lemma}
	$$\forall i \in S, \quad d'(i) \leq n$$
\end{lemma}

\underline{\textbf{Preuve :}}\\
L'ensemble des sommets dont la distance est inférieure à une valeur donnée est un sous-ensemble de
l'ensemble des sommets du graphe. Son cardinal est donc majoré par ce dernier.
Introduisons la fonction de potentiel $\Phi$ définie comme suit : $$
\Phi = \sum_{i : e(i) > 0} d'(i) / K $$

On peut remarquer qu'à n'importe quel moment de l'exécution de l'algorithme, $\Phi \geq 0$ et que
après l'initialisation : $Phi \leq n^2 / K$. Cette borne est atteinte dans le cas d'un graphe
complet puisque chaque n\oe ud ayant une distance de 1 et étant relié à la source, on obtient alors
$n$ n\oe uds actifs vérifiant tous $d'(n) = n$, d'où $\Phi = n^2/K$.

Regardons les effets des opérations de poussage et réétiquetage sur la fonction $\Phi$.

\begin{lemma}
	Une opération de réétiquetage augmente la valeur de $\Phi$ d'au plus $n/K$.
\end{lemma}

\underline{\textbf{Preuve :}}\\Dans le pire des cas, le sommet réétiqueté passe de la plus petite
distance du graphe à la plus grande. Une représentation d'un de ces cas est donnée \ref{pdc}. Il
s'agit là d'une chaîne dont la capacité de chacun des arcs est supérieure à la capacité du dernier
arc de la chaîne, de façon à ce que l'on observe un phénomène de flux et reflux influant sur la
distance de tous les sommets. Lors du dernier réétiquetage de l'algorithme on se retrouve dans le
cas représenté.
Aucun autre n\oe ud n'étant actif, on a, avant réétiquetage, $\Phi = d'(1) / K = 1/K$, or après
réétiquetage, le sommet $1$ est le sommet présentant la plus haute distance du graphe. On observe
dans ce cas une augmentation de $k/K$ soit $(n-2)/k$.

\begin{figure}
	\begin{center}
		\subfloat[Le graphe de flots]{
			\begin{tikzpicture}
				\tikzset{n/.style={draw=black, circle, minimum size=17pt},f/.style={->,>=latex}};
				\node[n] (s) at (0,0) {$s$};
				\node[n] (1) at (2,0) {$1$};
				\node[n] (2) at (4,0) {$2$};
				\node (pt) at (6,0) {$\dots$};
				\node[n] at (8,0) {$k$};
				\node[n] at (10,0) {$t$};

				\draw[f] (s) to node[above] (1) {$2/2$};
				\draw[f] (1) to node[above] (2) {$1/2$};
				\draw[f] (2) to node[above] (pt) {$1/2$};
				\draw[f] (pt) to node[above] (n) {$1/2$};
				\draw[f] (n) to node[above] (t) {$1/1$};
		\end{tikzpicture}}
		\subfloat[Excédents, distances et fonctions de potentiel]{
			\begin{tabular}{|c|c|c|c|} \hline
				$i$ & $d(i)$ & $e(i)$ & $d'(i)$ \\ \hline
				$s$ & $k+2$ & $\infty$ & $k+1$ \\ \hline
				$1$ & $k+1$ & $1$ & $1$ \\ \hline
				$2$ & $k+2$ & $0$ & $n+1$ \\ \hline
				$\dots$ & $\dots$ & $\dots$ & $\dots$ \\ \hline
				$n$ & $k+2$ & $0$ & $k+1$ \\ \hline
				$t$ & $0$ & $1$ & $0$ \\ \hline
				& & & $\Phi = 1/K$ \\ \hline
			\end{tabular}}
	\end{center}
	\label{pdc}
	\caption{Exemple de l'influence du réétiquetage sur la fonction $\Phi$}
\end{figure}

\begin{lemma}
	Une opération de poussage saturant augmente la valeur de $\Phi$ d'au plus $n/K$.
\end{lemma}

\underline{\textbf{Preuve :}}\\
Chaque opération de poussage non saturant rend actif, au plus un seul sommet supplémentaire,
appelons $k$ ce sommet. Or on sait que $d'(k) \leq n$, donc l'augmentation de $\Phi$ par l'ajout
d'un sommet actif est bornée par $n/K$.

\begin{lemma}
	Une opération de poussage non saturant n'augmente pas la valeur de $\Phi$.
\end{lemma}

\underline{\textbf{Preuve :}}\\
Soient $i$ et $j$ deux sommets tels qu'il est possible de réaliser un poussage non saturant de $i$
vers $j$.  Après cette opération, le sommet $i$ n'est plus actif et le sommet $j$ le devient, or
$d(i) = d(j) + 1$ donc $d'(i) > d'(j)$, on observe même une diminution de la valeur de $\Phi$ d'au
moins $1/K$.

Afin de quantifier le nombre de poussage non saturant, l'algorithme est scindée en phases définie
comme suit : une phase est l'ensemble des opérations de poussage ayant lieu entre deux changements
consécutifs de la valeur :
\begin{equation}
	d^* = \max\{d(v) : v \mbox{ est actif }\}
\end{equation}

Autrement dit il s'agit de l'ensemble des poussages ayant lieu
