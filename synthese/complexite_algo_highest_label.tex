Comme vu précédemment, l'algorithme High Label analyse les n\oe uds en fonction de la valeur de leur
fonction de distance. Cette sélection permet d'obtenir une borne maximale en $O(n^2\sqrt{m})$, c'est
ce que nous allons démontrer.

Soit une constante $K = \sqrt{m}$, soit la fonction $d': S \longrightarrow \mathbb{N}$, définie par
: $$
d'(i) = |\{j : d(j) \leq d(i) \}| \quad \forall i,j \in S $$
Autrement dit, $d'(i)$ représente le nombre de n\oe uds de $S$ étant plus proche du puits que $i$.

Soit la fonction de potentiel $\Phi$ définie comme suit : $$
\Phi = \sum_{i : e(i) > 0} d'(i) / K $$


