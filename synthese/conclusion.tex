Comme vu précédemment, les algorithmes de préflots sont des algorithmes très rapides pour la
résolution du problème de recherche de flot maximum. Et même s'il existe des méthodes dont la
complexité théorique est inférieure à la leur\footnote{On peut par exemple parler de l'algorihtme
pousser vers l'avant avec un arbre dynamique}, ils sont dans la pratique ceux offrant les meilleurs
résultats ainsi qu'une grande facilité de mise en \oe uvre\footnote{Relativement aux autres
algorithmes dont les performances sont à peu de choses près équivalentes}.

Mais au delà de cet aspect pratique, ils
représentent une approche intuitive des algorithmes de résolution de problème par relaxation de
contraintes \footnote{On peut par exemple citer dans cette famille, les méthodes arborescentes
telles que le \emph{Branch and Bound} ou \emph{Branch and Cut}}. Ils sont en effet le parfait
exemple de la puissance de ces méthodes, et mettent en avant combien il est coûteux de maintenir à
n'importe quelle étape de l'algorithme une solution réalisable.

